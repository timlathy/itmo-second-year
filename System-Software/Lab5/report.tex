\documentclass[listings]{labreport}
\departmentsubject{Кафедра вычислительной техники}{Системное программное обеспечение}
\titleparts{Лабораторная работа №5}{Основы регулярных выражений}
\students{Лабушев Тимофей Михайлович}
\usepackage{multicol}

\usepackage{etoolbox}
\makeatletter
\preto{\@verbatim}{\topsep=1pt \partopsep=1pt}
\makeatother

\begin{document}

\maketitlepage

\section*{Задание №1. Файл datebook}

\begin{enumerate}
\item Замените имя Jon на Jonathan:
\begin{verbatim}
sed 's/\<Jon\>/Jonathan/g' datebook
\end{verbatim}

\item Удалите первые три строки:
\begin{verbatim}
sed '1,3d' datebook
\end{verbatim}

\item Выведите строки с 5-ой по 10-ю:
\begin{verbatim}
sed -n '1,5p' datebook
\end{verbatim}

\begin{small}
  \verb|-n| отключает автоматический вывод строк — без этого ключа каждая из строк с 5 по 10 была бы напечатана дважды.
\end{small}

\item Удалите строки, содержащие Lane:
\begin{verbatim}
sed '/Lane/d' datebook
\end{verbatim}

\item Выведите все строки с с днем рождения в ноябре или декабре:
\begin{verbatim}
sed -nE '/^([^:]+:){3}(11|12)/p' datebook
\end{verbatim}

\begin{small}
  С ключом \verb|-E| используются расширенные регулярные выражения.
  Дата рождения находится в третьем поле. Поля отделяются сиволом \texttt{:}.
  Дата записана в американском формате, где первое число соответствует месяцу.
\end{small}

\item Добавить три символа * в конец строк, начинающихся с Fred:
\begin{verbatim}
sed '/^Fred/ s/$/***/' datebook
\end{verbatim}

\item Замените строку, содержащую Jose, на JOSE HAS RETIRED:
\begin{verbatim}
sed '/Jose/ s/.*/JOSE HAS RETIRED/' datebook
\end{verbatim}

\item Замените дату рождения Popeye на 11/14/46. При этом подразумевается, что вы не знаете значение даты, хранящейся в файле. Составьте и используйте регулярное выражение для ее поиска:
\begin{verbatim}
sed -E 's/^(Popeye([^:]+:){3})[^:]+/\111\/14\/46/' datebook
\end{verbatim}

\begin{small}
  \verb|\1| в шаблоне замены содержит первую группу, в которую входят все поля до даты.
\end{small}

\item Удалите все пустые строки:
\begin{verbatim}
sed '/^$/d' datebook
\end{verbatim}

\item Напишите программу на языке редактора sed, которая:
\begin{small}
\begin{enumerate}
\item вставляет перед первой строкой заголовок TITLE OF FILE
\item удаляет последнее поле, значение которого кратно 500
\item меняет местами имя и фамилию
\item добавляет к концу каждой строки фразу THE END
\end{enumerate}
\end{small}
\begin{verbatim}
> cat script.sed
1 i\TITLE OF FILE
s/:[0-9]*[05]00$//
s/^([^ ]+) ([^ :]+)/\2 \1/
s/$/THE END/
> sed -E -f script.sed datebook
\end{verbatim}

\end{enumerate}
\end{document}
