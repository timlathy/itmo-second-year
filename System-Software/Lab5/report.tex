\documentclass[listings]{labreport}
\departmentsubject{Кафедра вычислительной техники}{Системное программное обеспечение}
\titleparts{Лабораторная работа №5}{Основы регулярных выражений}
\students{Лабушев Тимофей Михайлович}
\usepackage{multicol}

\usepackage{etoolbox}
\makeatletter
\preto{\@verbatim}{\topsep=1pt \partopsep=1pt}
\makeatother

\begin{document}

\maketitlepage

\section*{Задание \#1. Файл datebook}

\begin{enumerate}
\item Замените имя Jon на Jonathan:
\begin{verbatim}
sed 's/\<Jon\>/Jonathan/g' datebook
\end{verbatim}

\item Удалите первые три строки:
\begin{verbatim}
sed '1,3d' datebook
\end{verbatim}

\item Выведите строки с 5-ой по 10-ю:
\begin{verbatim}
sed -n '1,5p' datebook
\end{verbatim}

\begin{small}
  \verb|-n| отключает автоматический вывод строк — без этого ключа каждая из строк с 5 по 10 была бы напечатана дважды.
\end{small}

\item Удалите строки, содержащие Lane:
\begin{verbatim}
sed '/Lane/d' datebook
\end{verbatim}

\item Выведите все строки с с днем рождения в ноябре или декабре:
\begin{verbatim}
sed -nE '/^([^:]+:){3}(11|12)/p' datebook
\end{verbatim}

\begin{small}
  С ключом \verb|-E| используются расширенные регулярные выражения.
  Дата рождения находится в третьем поле. Поля отделяются сиволом \texttt{:}.
  Дата записана в американском формате, где первое число соответствует месяцу.
\end{small}

\item Добавить три символа * в конец строк, начинающихся с Fred:
\begin{verbatim}
sed '/^Fred/ s/$/***/' datebook
\end{verbatim}

\item Замените строку, содержащую Jose, на JOSE HAS RETIRED:
\begin{verbatim}
sed '/Jose/ s/.*/JOSE HAS RETIRED/' datebook
\end{verbatim}

\item Замените дату рождения Popeye на 11/14/46. При этом подразумевается, что вы не знаете значение даты, хранящейся в файле. Составьте и используйте регулярное выражение для ее поиска:
\begin{verbatim}
sed -E 's/^(Popeye([^:]+:){3})[^:]+/\111\/14\/46/' datebook
\end{verbatim}

\begin{small}
  \verb|\1| в шаблоне замены содержит первую группу, в которую входят все поля до даты.
\end{small}

\item Удалите все пустые строки:
\begin{verbatim}
sed '/^$/d' datebook
\end{verbatim}

\item Напишите программу на языке редактора sed, которая:
\begin{small}
\begin{enumerate}
\item вставляет перед первой строкой заголовок TITLE OF FILE
\item удаляет последнее поле, значение которого кратно 500
\item меняет местами имя и фамилию
\item добавляет к концу каждой строки фразу THE END
\end{enumerate}
\end{small}
\begin{verbatim}
> cat script.sed
1 i\TITLE OF FILE
s/:[0-9]*[05]00$//
s/^([^ ]+) ([^ :]+)/\2 \1/
s/$/THE END/
> sed -E -f script.sed datebook
\end{verbatim}

\end{enumerate}

\section*{Задание \#2. Файл datafile}

\begin{enumerate}
\item \verb|sed '/north/p' datafile|

Строки, содержащие \verb|north|, встретятся в выводе дважды: каждая строка
печатается по умолчанию; команда \verb|p| также выводит строку.

\item \verb|sed -n '/north/p' datafile|

Ключ \verb|-n| отключает вывод строк по умолчанию, поэтому в выводе будут
только строки, содержащие \verb|north|.

\item \verb|sed '3d' datafile|

Число перед командой указывает на номер строки. Команда \verb|d| удаляет строку из вывода.

\item \verb|sed '3,$d' datafile|

С помощью запятой задается интервал — в данном случае от 3 строки до конца файла (\verb|$|).

\item \verb|sed '$d' datafile|

\verb|$| указывает на последнюю строку в файле.

\item \verb|sed '/north/d' datafile|

Строки, содержащие \verb|north|, исключаются из вывода.

\item \verb|sed 's/west/north/g' datafile|

Все вхождения \verb|west| в файле заменяются в выводе на \verb|north|.

\item \verb|sed -n 's/^west/north/p' datafile|

Строки, начинающиеся на \verb|west|, попадают в вывод (\verb|p|) с заменой \verb|west|
на \verb|north|. Остальное содержимое файла не выводится (\verb|-n|).

\item \verb|sed 's/[0-9][0-9]$/&.5/' datafile|

\verb|&| соответствует совпавшей части строки, поэтому в результате работы к строкам,
оканчивающимся на две цифры, будет дописано \verb|.5|.

\item \verb|sed -n 's/Hemenway/Jones/gp' datafile|

Будут выведены (\verb|p|) только (\verb|-n|) те строки, которые содержат \verb|Hemenway|,
с заменой всех вхождений \verb|Hemenway| на \verb|Jones|.

\item \verb|sed -n 's/\(Stag\)got/\1ianne/p' datafile|

Будут выведены только те строки, которые содержат \verb|Staggot|, с заменой первого
в строке вхождения на \verb|Stagianne| (\verb|\1| — первая группа).

\item \verb|sed 's#14#88#g' datafile|

При использовании команды замены любой символ (кроме \verb|\| и перехода на новую строку),
следующий за \verb|s|, принимается за разделитель — команда аналогична \verb|s/14/88/g|.

\item \verb|sed -n '/west/,/east/p' datafile|

В вывод попадут строки в диапазоне от первой, содержащей \verb|west|, до первой, содержащей
\verb|east|.

\item \verb|sed -n '5,/^northeast/p' datafile|

В вывод попадут строки c пятой по первую после пятой, начинающуюся на \verb|northeast|.

\item \verb|sed '/west/,/east/s/$/**WAKA**/' datafile|

К концу каждой строки из диапазона от первого вхождения \verb|west| до первого вхождения
\verb|east| будет добавлено \verb|**WAKA**|.

\item \verb|sed -e '1,3d' -e 's/Hemenway/Jones/' datafile|

Ключ \verb|-e| добавляет команду к выполняемым над исходными данными: в данном случае,
из вывода будут удалены с первой по третью строки файла, а также произведена
замена первых в каждой строке вхождений \verb|Hemenway| на \verb|Jones|.

\item \verb|sed '/Suan/r newfile' datafile|

После каждой строки, содержащей \verb|Suan|, будет добавлено содержание файла \verb|newfile|.

\item \verb|sed -n '/north/w newfile' datafile|

Вывод отсутствует, но каждая строка, содержащая \verb|north|, записывается в файл.

\item \begin{verbatim}
sed '/^north /a\
--->THE NORTH SALES DISTRICT HAS MOVED<---' datafile
\end{verbatim}

После каждой строки, начинающейся на \verb|north|, будет добавлена строка
\begin{verbatim}
--->THE NORTH SALES DISTRICT HAS MOVED<---
\end{verbatim}

\item \begin{verbatim}
sed '/eastern/i\
NEW ENGLAND REGION\
-------------------------------------' datafile
\end{verbatim}

Перед каждой строкой, содержащей \verb|eastern|, будет вставлено:
\begin{verbatim}
NEW ENGLAND REGION
-------------------------------------
\end{verbatim}

\item \begin{verbatim}
sed '/eastern/c\
THE EASTERN REGION HAS BEEN TEMPORARILY CLOSED' datafile
\end{verbatim}

Вместо строк, содержащих \verb|eastern|, будет выведено:
\begin{verbatim}
THE EASTERN REGION HAS BEEN TEMPORARILY CLOSED
\end{verbatim}

\end{enumerate}


\end{document}
