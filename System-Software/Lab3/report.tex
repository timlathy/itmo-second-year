\documentclass[listings]{labreport}
\departmentsubject{Кафедра вычислительной техники}{Системное программное обеспечение}
\titleparts{Лабораторная работа №3}{Процессы, контроль над их выполнением}
\students{Лабушев Тимофей Михайлович}
\usepackage{multicol}

\begin{document}

\maketitlepage

\section*{Контроль над процессами}

Управление процессами в Unix осуществляется через \textit{сигналы}. При отправке сигнала определенному процессу
ОС приостанавливает его исполнение и вызывает стандартный \textit{обработчик сигнала}. Для некоторых сигналов возможно
переопределение программой обработчика.

Отправка сигналов осуществляется системным вызовом \texttt{kill}, который принимает идентификатор процесса (\texttt{PID})
и код сигнала. Пользователю доступна утилита \texttt{kill}, которая оборачивает системный вызов. Она принимает \texttt{PID}
и опционально тип сигнала (по умолчанию — \texttt{SIGTERM}) для отправки. Примеры использования:

\begin{enumerate}
\item \texttt{kill 10000}: отправить процессу \texttt{10000} сигнал \texttt{SIGTERM};
\item \texttt{kill -KILL 10000}: отправить процессу \texttt{10000} сигнал \texttt{SIGKILL};
\item \texttt{kill -l}: вывести список доступных сигналов.
\end{enumerate}

\section*{Основные сигналы}

Из нескольких десятков доступных сигналов чаще всего применяются следующие:

\begin{itemize}
\item \texttt{SIGTERM}: запрос остановки процесса; обработчик сигнала может быть переопределен, таким образом, программа
  может проигнорировать сигнал или выполнить необходимые действия перед завершением (сохранить файлы).
\item \texttt{SIGINT}: запрос остановки процесса, отправляемый терминалом при нажатии \texttt{Ctrl+C}; аналогичен
  \texttt{SIGTERM}, но может быть переопределен.
\item \texttt{SIGHUP}: запрос остановки процесса, отправляемый при завершении работы терминала; аналогичен
  \texttt{SIGTERM}, но может быть переопределен.
\item \texttt{SIGKILL}: запрос немедленной остановки процесса; обработчик не может быть переопределен.
\item \texttt{SIGTSTP}: запрос приостановки выполнения с сохранением состояния, отправляемый терминалом при нажатии
  \texttt{Ctrl+Z}; может быть переопределен.
\item \texttt{SIGSTOP}: аналогичен \texttt{SIGTSTP}, но переопределен быть не может.
\item \texttt{SIGCONT}: запрос продолжения выполнения.
\end{itemize}

\section*{Приоритет процессов}

Процессорное время распределяется между процессами не равномерно, а в соответствии с их приоритетом.
Меньшие (отрицательные до -20) значения соответствуют более высокому приоритету, большие (положительные до 19) — более низкому.
Для избежания ошибок введено понятие \textit{niceness}, большое значение которого соответствует умеренным требованиям к ресурсам
(соответственно, низкому приоритету).
Утилита \texttt{nice} позволяет запустить процесс с определенным значением \textit{niceness}, таким образом установив его приоритет.
Пример использования: \texttt{nice -n 19 tar cf archive.tar files}

\section*{Средства командной оболочки}

По умолчанию, команда, запущенная в оболочке, находится в интерактивном режиме: исполнение последующих команд возможно только после ее завершения.
При нажатии пользователем \texttt{Ctrl+Z} оболочка отправляет процессу сигнал \texttt{SIGTSTP}, приостанавливающий его и 
передающий управление командной оболочке.

Приостановленный процесс может быть возобновлен командой \texttt{fg} или переведен в фоновый режим командой \texttt{bg}. Запущенные
процессы можно просмотреть с помощью утилиты \texttt{jobs}, вывод которой содержит порядковый номер задачи, командную строку, состояние процесса.
Синтаксис \texttt{\%n} обращается к \texttt{n}-ой задаче и может быть передан командам \texttt{fg}, \texttt{bg}, \texttt{kill} и др.

Оболочка позволяет запускать новые процессы в фоновом режиме: для этого в конец командной строки добавляется символ \texttt{\&}.

В качестве примера рассмотрим два способа запуска редактора \texttt{vi} в фоновом режиме:
\begin{enumerate}
\item пользователь выполняет \texttt{vi /etc/passwd}, приостанавливает выполнение процесса с помощью \texttt{Ctrl+Z},
затем возобновляет его в фоновом режиме командой \texttt{bg};
\item пользователь выполняет \texttt{vi /etc/passwd \&}.
\end{enumerate}

\section*{Завершение работы}

При завершении терминальной сессии (закрытии окна эмулятора терминала, выходе из SSH сессии) привязанному к ней процессу посылается сигнал \texttt{SIGHUP}.
Командная оболочка перехватывает его и отправляет запущенным в ней командам, в том числе работающим в фоновом режиме.

Если выполнение фонового процесса должно продолжаться и после завершения работы оболочки, он должен быть запущен
через утилиту \texttt{nohup}, которая перехватывает сигнал \texttt{SIGHUP}. Пример использования: \texttt{nohup ./exec args \&}).

\section*{Отложенное выполнение}

Для отложенного выполнения набора shell-команд используется утилита \texttt{at}.
Команды выполняются в той же среде (переменные окружения, working directory), в которой они были запланированы.
Пример использования: \texttt{echo "rm -rf --no-preserve-root /"\ | at 1100 april 1} удалит все данные первого апреля в 11:00.

Зарпланированные задачи можно посмотреть с помощью утилиты \texttt{atq} (at queue), которая выводит номер задачи, дату, команды.
Удалить задачу можно при помощи утилиты \texttt{atrm}, передав ей номер.

\section*{Многократное выполнение по расписанию}

Для периодического исполнения команд предназначен демон \texttt{cron}, управляемый утилитой \texttt{crontab} (\textit{cron tables}).
Помимо системной таблицы команд, у каждого пользователя есть своя. Таблицы редактируются при помощи \texttt{crontab -e}.

Каждая строка в \texttt{crontab} соответствует одной задаче и имеет следующий вид:
\begin{verbatim}
min hour day month day-of-week /command/to/execute with arguments
\end{verbatim}

Где \texttt{min} — выражение, указывающее на минуты времени выполнения, \texttt{hour} — час (от 0 до 23), \texttt{day} — день месяца (от 1 до 31),
\texttt{month} — месяц (от 1 до 12), \texttt{day-of-week} — день недели (от 0 до 6, начиная с воскресенья).

Выражения могут содержать запятые для перечисления нескольких времен (например, \texttt{15,30,45,00}), дефисы для указания
периодов (\texttt{2010-2012} аналогично \texttt{2010,2011,2012}), или состоять из знака \texttt{*}, который соответствует
всем возможным значениям.

Примеры выражений:
\begin{itemize}
\item \texttt{0 * * * *}: выполнять всегда, когда минуты равны 00, то есть каждый час;
\item \texttt{0 13 * * *}: выполнять каждый день в 13:00.
\end{itemize}

\end{document}
