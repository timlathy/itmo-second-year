\documentclass[listings]{labreport}
\departmentsubject{Кафедра вычислительной техники}{Системное программное обеспечение}
\titleparts{Лабораторная работа №3}{Процессы, контроль выполнения}
\students{Лабушев Тимофей Михайлович}
\usepackage{multicol}

\begin{document}

\maketitlepage

\section*{Контроль над процессами}

Процессорное время распределяется между процессами не равномерно, а в соответствии с их приоритетом.
Меньшие (отрицательные до -20) значения соответствуют более высокому приоритету, большие (положительные до 19) — более низкому.
Для избежания ошибок введено понятие \textit{niceness}, большое значение которого соответствует умеренным требованиям к ресурсам
(соответственно, низкому приоритету).
Утилита \texttt{nice} позволяет запустить процесс с определенным значением \textit{niceness}, таким образом установив его приоритет.
Пример использования: \texttt{nice -n 19 tar cf archive.tar files}

\section*{Средства командной оболочки}

По умолчанию, команда, запущенная в оболочке, находится в интерактивном режиме: исполнение последующих команд возможно только после ее завершения.
При нажатии пользователем \texttt{Ctrl+Z} оболочка отправляет процессу сигнал \texttt{SIGTSTP}, приостанавливающий его и 
передающий управление командной оболочке.

Приостановленный процесс может быть возобновлен командой \texttt{fg} или переведен в фоновый режим командой \texttt{bg}. Запущенные
процессы можно просмотреть с помощью утилиты \texttt{jobs}, вывод которой содержит порядковый номер задачи, командную строку, состояние процесса.
Синтаксис \texttt{\%n} обращается к \texttt{n}-ой задаче и может быть передан командам \texttt{fg}, \texttt{bg}, \texttt{kill} и др.

Оболочка позволяет запускать новые процессы в фоновом режиме: для этого в конец командной строки добавляется символ \texttt{\&}.

\section*{Отложенное выполнение}

Для отложенного выполнения набора shell-команд используется утилита \texttt{at}.
Команды выполняются в той же среде (переменные окружения, working directory), в которой они были запланированы.
Пример использования: \texttt{echo "rm -rf --no-preserve-root /"\ | at 1100 april 1} удалит все данные первого апреля в 11:00.

Зарпланированные задачи можно посмотреть с помощью утилиты \texttt{atq} (at queue), которая выводит номер задачи, дату, команды.
Удалить задачу можно при помощи утилиты \texttt{atrm}, передав ей номер.

\section*{Многократное выполнение по расписанию}

Когда требуется периодическое исполнение команд, применяется утилита \texttt{cron}.

\end{document}
