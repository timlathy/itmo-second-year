\documentclass[listings]{labreport}
\departmentsubject{Кафедра вычислительной техники}{Системное программное обеспечение}
\titleparts{Лабораторная работа №6}{Основы awk}
\students{Лабушев Тимофей Михайлович}
\usepackage{multicol}

\usepackage{etoolbox}
\makeatletter
\preto{\@verbatim}{\topsep=1pt \partopsep=1pt}
\makeatother

\begin{document}

\maketitlepage

\section*{Задание \#1. Файл db}

\noindent Пример строки из файла:
\begin{verbatim}
Mike Harrington:(510) 548-1278:250:100:175
\end{verbatim}

\begin{enumerate}
\item Вывести все номера телефонов:
\begin{verbatim}
awk -F : '{ print $2 }' db
\end{verbatim}

\begin{small}
  \verb|$0| содержит всю строку, \verb|$1| — первое поле в строке и т.д.\\
  Разделитель полей устанавливается с помощью ключа \verb|-F|
\end{small}

\item Вывести номер телефона, принадлежащий сотруднику Dan:
\begin{verbatim}
awk -F : '/^Dan/ { print $2 }' db
\end{verbatim}

\item Вывести имя, фамилию и номер телефона сотрудницы Susan:
\begin{verbatim}
awk -F : '/^Susan/ { print $1, $2 }' db
\end{verbatim}

\begin{small}
  В выражении \verb|print| пробел между элементами вставляется с помощью \verb|,|
\end{small}

\item Вывести все фамилии, начинающиеся с буквы D:
\begin{verbatim}
awk -F '[ :]' '$2 ~ /^D/ { print $2 }' db
\end{verbatim}

\begin{small}
  Разделитель полей (\verb|-F|) задается регулярным выражением.
\end{small}

\item Вывести все имена, начинающиеся с буквы C или E:
\begin{verbatim}
awk '$1 ~ /^[CE]/ { print $1 }' db
\end{verbatim}

\begin{small}
  Здесь достаточно стандартного разделителя полей — последовательности
  из одного или более пробела или символа табуляции.
\end{small}

\item Вывести все имена, состоящие только из четырех букв:
\begin{verbatim}
awk '$1 ~ /^\w{4}$/ { print $1 }' db
\end{verbatim}

\item Вывести имена сотрудников, префикс номера телефона которых 916:
\begin{verbatim}
awk -F '[ :]' '$3 == "(916)" { print $1 }' db
\end{verbatim}

\item Вывести денежные вклады сотрудника Mike, предваряя каждую сумму знаком \verb|$|:
\begin{verbatim}
awk -F ':' '/^Mike/ { printf "$%s, $%s, $%s\n", $3, $4, $5 }' db
\end{verbatim}

\item Вывести инициалы всех сотрудников:
\begin{verbatim}
awk -F '[ :]' '{ print substr($1, 0, 1), substr($2, 0, 1) }' db
\end{verbatim}

\item Создать командный файл awk, который: 
\begin{enumerate}
\item печатает полные имена и номера телефонов всех сотрудников по фамилии Savage
\item печатает денежные вклады сотрудника по имени Chet
\item печатает сотрудников, денежные вклады которых в первом месяце составили 250\$
\item подсчитывает сумму вкладов за каждый месяц в отдельности и вывести это в виде оформленной таблицы
\item подсчитывает средний вклад за каждый месяц и выводит результаты округлённо до второго знака после запятой
\item в конце выводит текущее время и результат выполнения команды ls
\end{enumerate}

\begin{verbatim}
BEGIN {
  FS = ":";
  IGNORED_FIELDS = 2;
}
{
  if ($1 ~ /Savage$/)
    savage_contacts[NR] = $1 " " $2;

  empl_stats[NR]["name"] = $1;
  for (i = 1 + IGNORED_FIELDS; i <= NF; i++) {
    empl_stats[NR]["deposits"][i - IGNORED_FIELDS] = $(i);
    empl_stats[NR]["deposits"]["total"] += $(i);
    monthly_deposits[i - IGNORED_FIELDS] += $(i);
  }
  months = NF - IGNORED_FIELDS;
}
END {
  print "a) Savage' contact data";
  for (i in savage_contacts)
    print savage_contacts[i];
  print "b) Chet's deposits:";
  for (i = 1; i <= NR; i++) {
    if (empl_stats[i]["name"] ~ /Chet/)
      for (m = 1; m <= months; m++)
        print empl_stats[i]["deposits"][m];
  }
  print "c) $250 first month deposits:";
  for (i = 1; i <= NR; i++) {
    if (empl_stats[i]["deposits"][1] == 250)
      print empl_stats[i]["name"];
  }
  print "d) Total monthly deposits:";
  printf("month 1");
  for (m = 2; m <= months; m++) printf(" | month %d", m);
  printf("\n");
  printf("%7d", monthly_deposits[1]);
  for (m = 2; m <= months; m++) printf(" | %7d", monthly_deposits[m]);
  printf("\n");
  print "e) Average monthly deposit:";
  printf("month 1");
  for (m = 2; m <= months; m++) printf(" | month %d", m);
  printf("\n");
  printf("%7.2f", monthly_deposits[1] / NR);
  for (m = 2; m <= months; m++) printf(" | %7.2f", monthly_deposits[m] / NR);
  printf("\n");
  print "f) Current time, directory contents";
  system("date +%R");
  system("ls");
}
\end{verbatim}

\end{enumerate}

\section*{Задание \#2}

\verb|nawk| — AT\&T версия утилиты \verb|awk|. В современных дистрибутивах Linux
встречается GNU версия утилиты \verb|awk|, \verb|gawk|.

\begin{enumerate}
\item \verb|nawk '/west/' datafile|\\
Выводятся строки, содержащие последовательность \verb|west|.

\item \verb|nawk '/^north/' datafile|\\
Выводятся строки, начинающиеся с последовательности \verb|north|.

\item \verb-nawk '/^(no|so)/' datafile-\\
Выводятся строки, начинающиеся с \verb|no| или \verb|so|.

\item \verb|nawk '{print $3, $2}' datafile|\\
Через пробел выводятся сначала третье, затем второе поле каждой строки.

\item \verb|nawk '{print $3 $2}' datafile|\\
Слитно выводятся снала третье, затем второе поле каждой строки.

\item \verb|nawk '{print $0}' datafile|\\
Выводится каждая строка, то есть содержимое файла.

\item \verb|nawk '{print "Number of fields: "NF}' datafile|\\
Для каждой строки выводится строка Number of fields: $n$, где $n$ равен числу полей в строке.

\item \verb|nawk '/northeast/{print $3, $2}' datafile|\\
Через пробел выводятся третье и второе поля строк, содержащих последовательность \verb|northeast|.

\item \verb|nawk '/E/' datafile|\\
Выводятся строки, содержащие \verb|E|.

\item \verb|nawk '/^[ns]/{print $1}' datafile|\\
Выводится первое поле каждой строки, начинающейся с \verb|n| или \verb|s|.

\item \verb|nawk '$5 ~ /\.[7-9]+/' datafile|\\
Выводятся строки, в которых пятое поле удовлетворяет шаблону. Шаблон определяет, что
в строке должна быть точка, за которой следует одна или несколько цифр \verb|8|, \verb|9|, \verb|10|.

\item \verb|nawk '$2 !~ /E/{print $1, $2}' datafile|\\
Для строк, второе поле которых не содержит \verb|E|, через пробел выводятся первое и второе поле.

\item \verb|nawk '$3 ~ /^Joel/{print $3 " is a nice guy."}' datafile|\\
Для строк, третье поле которых начинается на \verb|Joel|, выводится сообщение \verb|$3 is a nice guy.|, где
\verb|$3| — значение третьего поля.

\item \verb|nawk '$8 ~ /[1-9][0-2]$/{print $8}' datafile|\\
Для строк, восьмое поле которых заканчивается на цифру от \verb|1| до \verb|9| и следующую за ней цифру \verb|0|, \verb|1| или \verb|2|,
выводится восьмое поле.

\item \verb|nawk '$4 ~ /Chin$/{print "The price is $" $8 "."}' datafile|\\
Для строк, четвертое поле которых заканчивается на \verb|Chin|, выводится сообщение \verb|The price is $[].|, где \verb|[]| — восьмое поле.

\item \verb|nawk '/TJ/{print $0}' datafile|\\
Выводятся строки, содержащие \verb|TJ|.

\end{enumerate}

\end{document}
