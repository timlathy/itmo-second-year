\documentclass[listings]{labreport}
\departmentsubject{Кафедра вычислительной техники}{Системное программное обеспечение}
\titleparts{Лабораторная работа №1}{Введение в UNIX}
\students{Лабушев Тимофей Михайлович}

\begin{document}

\maketitlepage

\section*{Базовые команды}

\texttt{ls} (сокр. \textit{list}): отражает имена и атрибуты файлов.
Без аргументов показывает содержание текущей директории. Принимает как аргумент путь к директории или файлу,
информация о которых будет отображена.

\texttt{pwd} (сокр. \textit{print working directory}): выводит полный путь к текущей рабочей директории.

\texttt{cd} (сокр. \textit{change directory}): изменяет текущую рабочую директорию на путь, переданный в аргументе.
Без аргументов выполняет переход в домашнюю директорию (\texttt{\$HOME}).

\texttt{rm} (сокр. \textit{remove}): удаляет файлы и директории, пути к которым передаются в аргументах.

\texttt{mv} (сокр. \textit{move}): перемещает файлы. Если исходный файл и место назначения
располагаются в одной файловой системе, то выполняется переименование (т.е. изменение метаданных без доступа к самим данным).
В противном случае производится копирование в место назначения и удаление исходного файла. Распознает следующие комбинации аргументов:

\begin{enumerate}
\item {[исходный путь]} [путь к существующей директории, в которую будет перемещен файл]
\item {[исходный путь]} [путь к файлу, в который будет переименован исходный]
\item {[несколько исходных путей]} [путь к существующей директории, в которую будут перемещены все файлы]
\end{enumerate}

\texttt{cp} (сокр. \textit{copy}): копирует файлы. Комбинации аргументов сходны принимаемым командой \texttt{mv}; если исходный путь
является директорией, то необходим дополнительный флаг \texttt{-r} для рекурсивного копирования ее содержимого.

\texttt{mkdir} (сокр. \textit{make directory}): создает директории, пути к которым передаются в аргументах, если последние
не указывают на уже существующие файлы.

\texttt{rmdir} (сокр. \textit{remove directory}): удаляет пустые директории.

\texttt{type}: выводит интерпретацию аргументов как команд. Существует несколько типов команд:

\begin{enumerate}
\item \textit{builtin}: встроена в командную оболочку — код содержится в интерпретаторе, при исполнении не создается отдельный процесс
\item \textit{alias}: заменяется текстовой подстановкой (см. \texttt{alias})
\item \textit{file}: указывает на исполняемая файл (\texttt{type} печатает полный путь)
\item \textit{keyword}: является ключевым словом интерпретатора
\item \textit{function}: задана функцией на языке сценариев интерпретатора
\end{enumerate}

\texttt{file}: выводит тип файла, определяя его серией проверок. В частности, считываются сигнатуры бинарных форматов
(байты определенного значения, расположенные по заданному отступу — "\textit{magic number}"),
производится текстовой поиск конструкций некоторых языков программирования.

\texttt{find}: производит файловый поиск по выражениям, которые проверяют название, тип и другие метаданные.
Для каждого результата выполняются заданные действия: выводится путь, вызывается утилита с путем к файлу в аргументах, файл удаялется и т.д.

\end{document}
