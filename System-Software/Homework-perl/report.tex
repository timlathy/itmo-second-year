\documentclass[listings]{labreport}
\departmentsubject{Кафедра вычислительной техники}{Системное программное обеспечение}
\titleparts{Самостоятельная работа №2}{Знакомство с Perl}
\students{Лабушев Тимофей Михайлович}

\begin{document}

\maketitlepage

\section*{Регулярные выражения}

\begin{itemize}
\item \verb|perl -e '$var = "abc123"; print ($var =~ /./ or 0);'|

Результат применения выражения — \texttt{1},
поскольку ему соответствует любая непустая строка.

\item \verb|perl -e '$var = "abc123"; print ($var =~ /[A-Z]*^/ or 0);'|

Результат = \texttt{1}, поскольку \verb|^| означает начало строки, а \verb|*|
указывает на то, что заглавная буква может не встретиться до начала строки.

\item \verb|perl -e '$var = "abc123"; print ($var =~ /(\d)2(\1)/ or 0);'|

Результат = \texttt{0}, поскольку \verb|\1| ссылается на первую группу,
которая, в свою очередь, выбирает одну цифру. Выражению соответствует
любоая строка, содержащая две одинаковые цифры, между которыми помещена цифра 2.

\item \verb|perl -e '$var = "abc123"; print ($var =~ /abc$/ or 0);'|

Результат = \texttt{0}, поскольку выражению соответствуют лишь те строки,
которые заканчиваются последовательностью \texttt{abc}.

\item \verb|perl -e '$var = "abc123"; print ($var =~ /1234?/ or 0);'|

Результат = \texttt{1}. Выражению соответствует любая строка, содержащая
\texttt{123} или \texttt{1234}.
\end{itemize}

\end{document}
