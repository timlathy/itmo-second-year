\documentclass[listings]{labreport}
\departmentsubject{Кафедра вычислительной техники}{Системное программное обеспечение}
\titleparts{Самостоятельная работа №1}{Знакомство с редактором vim}
\students{Лабушев Тимофей Михайлович}
\usepackage{multicol}

\begin{document}

\maketitlepage

\section*{Введение. Редакторы vi и vim}

Отличительной чертой \texttt{vi} среди текстовых редакторов является модальность:
наличие нескольких режимов работы, в которых различается назначение клавиш
и, как следствие, возможные действия.

Редактор \texttt{vim} является развитием \texttt{vi} и включает в себя
множество возможностей, которые встречаются в современных редакторах, но
отсутствуют в \texttt{vi}: подсветка синтаксиса, язык скриптов и сторонние расширения,
разделение окна на панели и владки.

\section*{Режимы}

Как обозначено выше, определенные действия по редактированию текста
выполняются в определенных режимах. Основными являются:

\begin{enumerate}
\item \textit{командный (command)}, также называемый \textit{нормальным (normal)}:
  вводимые символы интерпретируются как команды, которыми осуществляется навигация
  и производятся другие действия над текстом;
\item \textit{ввода (insert)}: вводимые символы вставляются в текст;
\item \textit{командной строки (command-line)}: вводимые символы формируют
  командную строку для управления редактором (открытие и закрытие файлов,
  установка параметров).
\end{enumerate}

В дополнение к ним, \texttt{vim} включает в себя \textit{визуальный (visual)} режим
для выделения текста. Команды нормального режима, применяемые к одному символу или линии,
будут применены ко всей выделенной области. 

\end{document}
