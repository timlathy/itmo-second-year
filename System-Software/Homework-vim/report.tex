\documentclass[listings]{labreport}
\departmentsubject{Кафедра вычислительной техники}{Системное программное обеспечение}
\titleparts{Самостоятельная работа №1}{Знакомство с редактором vim}
\students{Лабушев Тимофей Михайлович}
\usepackage{multicol}

\begin{document}

\maketitlepage

\section*{Введение. Редакторы vi и vim}

Отличительной чертой \texttt{vi} среди текстовых редакторов является модальность:
наличие нескольких режимов работы, в которых различается назначение клавиш
и, как следствие, возможные действия.

Редактор \texttt{vim} является развитием \texttt{vi} и включает в себя
множество возможностей, которые встречаются в современных редакторах, но
отсутствуют в \texttt{vi}: подсветка синтаксиса, язык скриптов и сторонние расширения,
разделение окна на панели и владки.

\section*{Режимы}

Как обозначено выше, определенные действия по редактированию текста
выполняются в определенных режимах. Основными являются:

\begin{itemize}
\item \textit{командный (command)}, также называемый \textit{нормальным (normal)}:
  вводимые символы интерпретируются как команды, которыми осуществляется навигация
  и производятся другие действия над текстом;
\item \textit{ввода (insert)}: вводимые символы вставляются в текст;
\item \textit{командной строки (command line)}: вводимые символы формируют
  командную строку для управления редактором (открытие и закрытие файлов,
  установка параметров).
\end{itemize}

В дополнение к ним, \texttt{vim} включает в себя \textit{визуальный (visual)} режим
для выделения текста. Команды нормального режима, применяемые к одному символу или линии,
будут применены ко всей выделенной области. 

\section*{Набор текста}

Ввод текста осуществляется в режиме \textit{insert}, переход в который осуществляется
следующими командами:

\begin{itemize}
\item \texttt{i}: вставить перед символом, на котором остановлен курсор;
\item \texttt{a}: вставить после символа, на котором остановлен курсор;
\item \texttt{I}: вставить в начало строки;
\item \texttt{A}: вставить в конец строки;
\item \texttt{o}: начать ввод на новой строке, вставленной после позиции курсора;
\item \texttt{O}: начать ввод на новой строке, вставленной перед позицией курсора.
\end{itemize}

Переход из режима \textit{insert} в \textit{normal} осуществляется с помощью
\texttt{ESC}.

\section*{Открытие и запись файлов}

По двоеточию \texttt{:}, нажатому в режиме \textit{normal}, происходит
переход к командной строке, в которую вводятся различные служебные команды,
в том числе для работы с файлами. Для выполнения базовых операций достаточно
следующих:

\begin{itemize}
\item \texttt{:w}, \texttt{:write}: записать изменения в файл;
\item \texttt{:w <file>}, \texttt{:write <file>}: создать новый файл
  по указанному пути и записать в него текст;
\item \texttt{:w! <file>}, \texttt{:write! <file>}: создать новый файл
  \textit{или перезатереть существующий} по указанному пути и записать в него текст;
\item \texttt{:e}, \texttt{:edit}: считать редактируемый файл с диска (чтобы
  отразить изменения, сделанные вне редактора), если в текущем нет несохраненных изменений;
\item \texttt{:e <file>}, \texttt{:edit <file>}: открыть для редактирования
  указанный файл, если в текущем нет несохраненных изменений;
\item \texttt{:e!}, \texttt{:e! <file>}, \texttt{:edit!}, \texttt{:edit! <file>}:
  считать файл, не сохраняя текущих изменений;
\item \texttt{:r <file>}, \texttt{:read <file>}: вставить текст из файла под курсором.
\end{itemize}

В отличии от большинства современных редакторов с заметной задержкой при запуске,
\texttt{vim} открывается достаточно быстро для стиля работы "запустить
редактор, отредактировать файл, закрыть редактор". Имя файла, который будет создан или изменен,
передается аргументом командной строки: \texttt{vim <file>}.

\end{document}
