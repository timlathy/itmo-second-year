\documentclass[listings]{labreport}
\departmentsubject{Кафедра вычислительной техники}{Системное программное обеспечение}
\titleparts{Лабораторная работа №4}{Основы регулярных выражений}
\students{Лабушев Тимофей Михайлович}
\usepackage{multicol}

\begin{document}

\maketitlepage

\section*{Задание №1}

\begin{enumerate}
\item Вывести все строки, содержащие строку Sun:
\begin{verbatim}
grep Sun /usr/local/regexp/datebook
\end{verbatim}

\item Вывести все строки, где имена людей начинаются с J:
\begin{verbatim}
grep '^J' /usr/local/regexp/datebook
\end{verbatim}

\item Вывести все строки, заканчивающиеся на 700:
\begin{verbatim}
grep '700$' /usr/local/regexp/datebook
\end{verbatim}

\item Вывести все строки, которые не содержат 834:
\begin{verbatim}
grep -v 834 /usr/local/regexp/datebook
\end{verbatim}

\item Вывести все строки, содержащие людей с днем рождения в декабре:
\begin{verbatim}
grep -E "^([^:]+:){3}12/" /usr/local/regexp/datebook
\end{verbatim}

\begin{small}
  Дата рождения находится в третьем поле. Поля отделяются сиволом \texttt{:}.
  Дата записана в американском формате, где первое число — месяц, числа разделены \texttt{/}.
  \vspace{2mm}
\end{small}

\item Вывести все строки с телефонными номерами, начинающимися с 408:
\begin{verbatim}
grep "408-[0-9]\{3\}-[0-9]\{4\}" /usr/local/regexp/datebook
\end{verbatim}

\begin{small}
  Формат телефонных номеров: \texttt{ddd-ddd-dddd}, где \texttt{d} — цифра.
  \vspace{2mm}
\end{small}

\item Вывести все строки, содержащие последовательность символов из
  заглавной буквы, четырех строчных букв, запятой, пробела и одной заглавной буквы:
\begin{verbatim}
grep -E "[A-Z][a-z]{4},[ ][A-Z]" /usr/local/regexp/datebook
\end{verbatim}

\item Вывести все строки, в которых фамилия начинается с K или k:
\begin{verbatim}
grep -i "^\w\+ k" /usr/local/regexp/datebook
\end{verbatim}

\item Вывести все строки с их порядковыми номерами, где последнее числовое поле записи состоит из шести цифр:
\begin{verbatim}
grep -nE ":[0-9]{6}$" /usr/local/regexp/datebook
\end{verbatim}

\item Вывести все строки, содержащие слова Lincoln или lincoln:
\begin{verbatim}
grep "[Ll]incoln" /usr/local/regexp/datebook
\end{verbatim}
\end{enumerate}

\section*{Задание 2}

С точки зрения интепретатора командной оболочки, \verb|grep regex file| —
вызов утилиты \verb|grep| с двумя аргументами.
Первый аргумент интерпретируется утилитой как регулярное выражение,
второй как путь к файлу, по которому необходимо произвести поиск.

\begin{enumerate}
\item \verb|grep '\<Tom\>' db|

Символ \verb|\<| соответствует началу слова, \verb|\>| — концу слова.
Cлова разделяются любым символом кроме буквы, цифры, нижнего подчеркивания.
Соответственно, с регулярным выражением совпадут строки, включающие последовательность \texttt{Tom}
как обособленное слово.

\item \verb|grep 'Tom Savage' db|

Выражению соответствуют строки, содержащая последовательность \texttt{Tom Savage}.

\item \verb|grep '^Tommy' db|

Выражению соответствуют строки, начинающаяся (\verb|^|) с последовательности \texttt{Tommy}.

\item \verb|grep '\.bak$' db|

Выражению соответствуют строки, оканчивающиеся (\verb|$|) на \texttt{.bak}.

Стоит отметить, что \verb|.| интерпретируется как специальный символ в регулярных выражениях,
если ему не предшествует символ экранирования \verb|\|, как в данном случае.

\item \verb|grep '[Pp]yramid' *|

На месте \verb|[Pp]| может встретиться любой символ из набора \texttt{Pp}, поэтому
будут выбраны строки, содержащие \texttt{Pyramid} или \texttt{pyramid}.

\texttt{*} воспринимается интерпретатором командной оболочки как метасимвол. На его
место будут подставлены все файлы из текущей директории. В начале каждой найденной строки
\texttt{grep} выведет имя файла, в котором она находится.

\item \verb|grep '[A-Z]' db|

\verb|[A-Z]| соответствует любой заглавной букве (от A до Z), то есть будут выбраны все
строки, содержащие хотя бы одну букву.

\item \verb|grep '[0-9]' db|

\verb|[0-9]| соответствует любой цифре, то есть будут выбраны все строки, содержащие хотя бы одну цифру.

\item \verb|grep '[A-Z]...[0-9]' db|

Выражению соответствуют строки, в которых после одной из заглавных букв следует три любых символа,
затем цифра.

\end{enumerate}

Дополнительные ключи утилиты \verb|grep| позволяют изменить действие выражений:

\begin{enumerate}
\setcounter{enumi}{8}

\item \verb|grep -w '[tT]est' db|
\end{enumerate}


\end{document}
