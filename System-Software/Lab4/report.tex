\documentclass[listings]{labreport}
\departmentsubject{Кафедра вычислительной техники}{Системное программное обеспечение}
\titleparts{Лабораторная работа №4}{Основы регулярных выражений}
\students{Лабушев Тимофей Михайлович}
\usepackage{multicol}

\usepackage{etoolbox}
\makeatletter
\preto{\@verbatim}{\topsep=1pt \partopsep=1pt}
\makeatother

\begin{document}

\maketitlepage

\section*{Задание №1}

\begin{enumerate}
\item Вывести все строки, содержащие строку Sun:
\begin{verbatim}
grep Sun /usr/local/regexp/datebook
\end{verbatim}

\item Вывести все строки, где имена людей начинаются с J:
\begin{verbatim}
grep '^J' /usr/local/regexp/datebook
\end{verbatim}

\item Вывести все строки, заканчивающиеся на 700:
\begin{verbatim}
grep '700$' /usr/local/regexp/datebook
\end{verbatim}

\item Вывести все строки, которые не содержат 834:
\begin{verbatim}
grep -v 834 /usr/local/regexp/datebook
\end{verbatim}

\item Вывести все строки, содержащие людей с днем рождения в декабре:
\begin{verbatim}
grep -E "^([^:]+:){3}12/" /usr/local/regexp/datebook
\end{verbatim}

\begin{small}
  Дата рождения находится в третьем поле. Поля отделяются сиволом \texttt{:}.
  Дата записана в американском формате, где первое число — месяц, числа разделены \texttt{/}.
\end{small}

\item Вывести все строки с телефонными номерами, начинающимися с 408:
\begin{verbatim}
grep "408-[0-9]\{3\}-[0-9]\{4\}" /usr/local/regexp/datebook
\end{verbatim}

\begin{small}
  Формат телефонных номеров: \texttt{ddd-ddd-dddd}, где \texttt{d} — цифра.
\end{small}

\item Вывести все строки, содержащие последовательность символов из
  заглавной буквы, четырех строчных букв, запятой, пробела и одной заглавной буквы:
\begin{verbatim}
grep -E "[A-Z][a-z]{4},[ ][A-Z]" /usr/local/regexp/datebook
\end{verbatim}

\item Вывести все строки, в которых фамилия начинается с K или k:
\begin{verbatim}
grep -i "^\w\+ k" /usr/local/regexp/datebook
\end{verbatim}

\item Вывести все строки с их порядковыми номерами, где последнее числовое поле записи состоит из шести цифр:
\begin{verbatim}
grep -nE ":[0-9]{6}$" /usr/local/regexp/datebook
\end{verbatim}

\item Вывести все строки, содержащие слова Lincoln или lincoln:
\begin{verbatim}
grep "[Ll]incoln" /usr/local/regexp/datebook
\end{verbatim}
\end{enumerate}

\newpage
\section*{Задание 2}

С точки зрения интепретатора командной оболочки, \verb|grep regex file| —
вызов утилиты \verb|grep| с двумя аргументами.
Первый аргумент интерпретируется утилитой как регулярное выражение,
второй как путь к файлу, по которому необходимо произвести поиск.

Результат работы утилиты можно определить по коду возврата: он равен нулю,
если выражению соответствует хотя бы одна строка, иначе единице. В командной оболочке
код возврата последней выполненной команды хранится в переменной окружения \verb|$?|.

\begin{enumerate}
\item \verb|grep '\<Tom\>' db|

Символ \verb|\<| соответствует началу слова, \verb|\>| — концу слова.
Cлова разделяются любым символом кроме буквы, цифры, нижнего подчеркивания.
Соответственно, с регулярным выражением совпадут строки, включающие последовательность \texttt{Tom}
как обособленное слово.

\item \verb|grep 'Tom Savage' db|

Выражению соответствуют строки, содержащие последовательность \texttt{Tom Savage}.

\item \verb|grep '^Tommy' db|

Выражению соответствуют строки, начинающиеся (\verb|^|) с последовательности \texttt{Tommy}.

\item \verb|grep '\.bak$' db|

Выражению соответствуют строки, оканчивающиеся (\verb|$|) на \texttt{.bak}.

Стоит отметить, что \verb|.| интерпретируется как специальный символ в регулярных выражениях,
если ему не предшествует символ экранирования \verb|\|, как в данном случае.

\item \verb|grep '[Pp]yramid' *|

На месте \verb|[Pp]| может встретиться любой символ из набора \texttt{Pp}, поэтому
будут выбраны строки, содержащие \texttt{Pyramid} или \texttt{pyramid}.

\texttt{*} воспринимается интерпретатором командной оболочки как метасимвол. На его
место будут подставлены все файлы из текущей директории. В начале каждой найденной строки
\texttt{grep} выведет имя файла, в котором она находится.

\item \verb|grep '[A-Z]' db|

\verb|[A-Z]| соответствует любой заглавной букве (от A до Z), то есть будут выбраны все
строки, содержащие хотя бы одну букву.

\item \verb|grep '[0-9]' db|

\verb|[0-9]| соответствует любой цифре, то есть будут выбраны все строки, содержащие хотя бы одну цифру.

\item \verb|grep '[A-Z]...[0-9]' db|

Выражению соответствуют строки, в которых после одной из заглавных букв следует три любых символа,
затем цифра.

\end{enumerate}

\newpage
Дополнительные ключи утилиты \verb|grep| позволяют изменить действие выражений:

\begin{enumerate}
\setcounter{enumi}{8}

\item \verb|grep -w '[tT]est' db|

При указании ключа \texttt{-w} в строке, чтобы она была выбрана, выражению должно соответствовать целое слово.
Слово — последовательность букв, цифр, нижних подчеркиваний.

Таким образом, в выводе утилиты будут только те строки, в которых встречаются \texttt{test} или \texttt{Test}, отделенные
любым символом, не входящим в обозначенные выше.

\item \verb|grep -s 'Mark Todd' db|

С ключом \verb|-s| утилита \texttt{grep} не выводит сообщения об ошибках (при ошибке доступа к файлу). 

\item \verb|grep -v 'Mary' db|

Ключ \verb|-v| инвертирует поведение выражения: в вывод попадают строки, ему несоответствующие.
В данном примере из файла \texttt{db} будут взяты строки, которые не содержат последовательность \texttt{Mary}.

\item \verb|grep -i 'sam' db|

Ключ \verb|-i| применяет выражение без учёта регистра: из файла будут выбраны строки, которые содержат
\texttt{sam}, \texttt{Sam}, \texttt{SAM}, \texttt{sAm} и т.д.

\item \verb|grep -l 'Dear Boss' *|

С ключом \verb|-l| утилита выводит не каждую строку, соответствующую выражению, а только имена файлов,
в которых было найдено совпадение. В выводе команды будут имена файлов в текущей директории,
в которых встречается последовательность \texttt{Dear Boss}.

\item \verb|grep -n 'Tom' db|

С ключом \verb|-n| перед каждой строкой, соответствующей выражению, будет выведен ее номер в файле.
Номера строк начинаются с 1.

\end{enumerate}

При использовании метасимволов в регулярных выражениях необходимо учитывать их значение для командной оболочки.
Команда \verb|grep "$name" db|, к примеру, интерпретируется как вызов утилиты \texttt{grep} с первым аргументом
равным значению переменной окружения \verb|$name|. Чтобы подстановки внутри выражения не выполнялись, его можно
заключить в одинарные кавычки: \verb|grep '$name' db|.

\begin{enumerate}
\setcounter{enumi}{14}

\item \verb|grep "$name" db|

Из файла \texttt{db} будут выбраны строки, содержащие выражение из переменной окружения \verb|name|.

\item \verb|grep '$5' db|

Выражению соответствуют строки, содержащие \verb|$5|.

\begin{small}
Хотя \verb|$| обычно указывает на конец строки в регулярном выражении, при наличии символов после него он
воспринимается \texttt{grep} буквально, то есть как \verb|$|. 
\end{small}

\item \verb=ps -ef| grep '^ *user1'=

В первой колонке вывода \verb|ps -ef| находится пользователь, которому принадлежит процесс.
\verb|^ *| соответствует нулю или больше символов пробела в начале строки, \texttt{user1}
выберет процессы, имя владельца которых начинается с \texttt{user1}.

\end{enumerate}

Утилита \texttt{egrep} функционально соответствует \texttt{grep} с ключом \texttt{-E}.
Выражения, переданные такой команде, обрабатываются по несколько измененным (расширенным) правилам,
с дополнительными синтаксическими конструкциями.

\begin{enumerate}
\setcounter{enumi}{17}

\item \verb|egrep '^ +' db|

Выражению соответствуют строки, начинающиеся хотя бы с одного символа пробела.

\item \verb|egrep '^ *' db|

Выражению соответствуют ноль или больше символов пробела в начале строки.

\item \verb=egrep '(Tom|Dan) Savage' db=

Синтаксис \verb|()| описывает группу последовательностей. Оператор \verb=|= (или) указывает на то,
что любая из последовательностей может быть встречена на месте группы.
Соответственно, выражению соответствуют строки, содержащие \texttt{Tom Savage} или \texttt{Don Savage}.

\item \verb|egrep '(ab)+' db|

\verb|+| — метасимвол, означающий, что группа должна повториться один или несколько раз.
Выражение в примере эквивалентно \verb|ab|.

\item \verb|egrep '^X[0-9]?' db|

Выражению соответствуют строки, начинающиеся (\verb|^|) с \texttt{X},
за которым может следовать цифра. Выражение эквивалентно \verb|^X|
(отличие будет с ключом \texttt{--color}, который подсвечивает вхождения
и доступен в современных версиях утилиты).

\item \verb|egrep 'fun\.$' *|

Выражению соответствуют строки, оканчиваются на \texttt{fun.}. Поиск выполняется
по всем файлам в директории.

\item \verb|egrep '[A-Z]+' db|

Выражению соответствуют строки, содержащие хотя бы одну заглавную букву.

\item \verb|egrep '[0-9]' db|

Выражению соответствуют строки, содержащие хотя бы одну цифру.

\item \verb|egrep '[A-Z]...[0-9]' db|

Выражению соответствуют строки, содержащие последовательность из заглавной буквы,
трех любых символов, цифры.

\item \verb|egrep '[tT]est' db|

Выражению соответствуют строки, содержащие \texttt{Test} или \texttt{test}.

\item \verb=egrep '(Susan|Jean) Doe' db=

Аналогично примеру \textbf{20}, выражению соответствуют строки с
\texttt{Susan Doe} или \texttt{Jean Doe}.

\item \verb|egrep -v 'Mary' db|

Поведение аналогично примеру \textbf{11}.

\item \verb|egrep -i 'sam' db|

Поведение аналогично примеру \textbf{12}.

\item \verb|egrep -l 'Dear Boss' *|

Поведение аналогично примеру \textbf{13}.

\item \verb|egrep -n 'Tom' db|

Поведение аналогично примеру \textbf{14}.

\item \verb|egrep -s "$name" db|

Поведение аналогично примеру \textbf{15}, с тем отличием, что выражение
из переменной окружения \verb|$name| будет интепретироваться как
расширенное.
 
\end{enumerate}

\end{document}
