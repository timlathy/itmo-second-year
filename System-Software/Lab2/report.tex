\documentclass[listings]{labreport}
\departmentsubject{Кафедра вычислительной техники}{Системное программное обеспечение}
\titleparts{Лабораторная работа №2}{Создание справочных страниц}
\students{Лабушев Тимофей Михайлович}
\usepackage{multicol}

\begin{document}

\maketitlepage

\section*{Верстка страниц с применением roff}

На отдельных строках, начинающихся с \texttt{.}, указываются \textit{запросы (requests)} и
вызовы \textit{макросов}, аргументы которых разделены пробелом и опциально экранированны
двойными кавычками.

Макросы, обозначенные ниже, импортируются из пакета \texttt{man}, созданного специально
для написания справочных страниц.

\subsection*{Использованные директивы}

\texttt{.TH title section [extra1] [extra2] [extra3]}: устанавливает заголовок и секцию страницы.
Первый опциональный аргумент \texttt{extra*} помещается посередине последней строки страницы,
второй — в начале последней строки, третий — в заголовке.

Номер секции отражает категорию, под которую попадает описываемая команда или программа. Предопределены
следующие категории:

\begin{enumerate}
\item утилиты и команды оболочки
\item системные вызовы
\item библиотечные вызовы
\item устройства (доступные в разделе \texttt{/dev})
\item форматы файлов (например, \texttt{sshd\_config} для конфигурации OpenSSH)
\item игры
\item описания, соглашения и другая вспомогательная информация (например, \texttt{roff} для общего описания системы верстки roff)
\item команды администрирования, требующие прав суперпользователя (например, \texttt{fdisk} для управления разделами диска)
\end{enumerate}

\texttt{.SH heading}: печатает заголовок, устанавливая жирный шрифт большого размера.

\texttt{.SS subheading}: печатает подзаголовок.

\texttt{.TP}: сдвигает следующую за макросом строку влево, и с отступом от нее выравнивает последующий текст.
Используется в списке параметров, где слева располагается ключ, а справа — его эффект.

\texttt{.B text}: печатает содержание строки жирным шрифтом.

\texttt{.I text}: печатает содержание строки курсивом.

\subsection*{Управляющие последовательности}

В некоторых частях страницы требовалось использование нескольких стилей шрифта в одной строке, для чего применялись
управляющие последовательности \texttt{\textbackslash fB, \textbackslash fI, \textbackslash fP},
устанавливающие шрифт в жирный, курсив, или возвращающие его к прежнему значению соответственно.

\end{document}
