\documentclass[12pt, a4paper]{article}
\usepackage[a4paper, includeheadfoot, mag=1000,
            left=2cm, right=1.5cm, top=1.5cm, bottom=1.5cm,
            headsep=0.8cm, footskip=0.8cm]{geometry}
% Fonts
\usepackage{fontspec, unicode-math}
\setmainfont[Ligatures=TeX]{CMU Serif}
\setmonofont{CMU Typewriter Text}
\usepackage[english, russian]{babel}
% Indent first paragraph
\usepackage{indentfirst}
\setlength{\parskip}{5pt}
% Diagrams
\usepackage{graphicx}
\usepackage{floatrow}
% Python
\usepackage{pythontex}

\begin{document}
\begin{titlepage}

\noindent\textsc{ФГАОУ ВО «Санкт-Петербургский национальный исследовательский
университет информационных технологий, механики и оптики»\\[4mm]
Кафедра физики}
\vfill
\noindent\textbf{РАБОЧИЙ ПРОТОКОЛ И ОТЧЕТ\\[2mm]
ПО ЛАБОРАТОРНОЙ РАБОТЕ №5\\[4mm]
Изучение свободных затухающих электромагнитных колебаний}\\[16mm]
Преподаватель Ефремова Екатерина Александровна\\[2mm]
Студенты Лабушев Тимофей Михайлович, Нестеров Дали Константинович\\[2mm]
Группа P3202
\vfill
\noindent Санкт-Петербург\\[2mm]
2018 г.

\end{titlepage}

\section*{Введение}

\subsection*{Цель работы}

\subsection*{Требуемое оборудование и материалы}

\begin{enumerate}
\item Генератор напряжений ГН1
\item Осциллограф ОЦЛ2
\item Стенд с объектом исследования С3-ЭМ01
\item Проводники Ш4/Ш2 (5 шт), Ш2/Ш2 (3 шт), 2Ш4/BNC (2 шт)
\end{enumerate}

\subsection*{Рабочие формулы и исходные формулы}

\section*{Ход работы}

\subsection*{Исходные параметры}

\begin{pycode}
import numpy as np

L = 10 * 10**-3
C_1 = 0.022 * 10**-6
C_2 = 0.033 * 10**-6
C_3 = 0.047 * 10**-6
C_4 = 0.47 * 10**-6

def r3(f): return np.round(f, 3)
def r2(f): return np.round(f, 2)
\end{pycode}

\noindent
$$L = \py{L * 10**3} \textup{ мГн} \pm 10\%$$
$$\ C_1 = \py{r3(C_1 * 10**6)} \textup{ мкФ} \pm 10\%
,\ C_2 = \py{r3(C_2 * 10**6)} \textup{ мкФ} \pm 10\%
,\ C_3 = \py{r3(C_3 * 10**6)} \textup{ мкФ} \pm 10\%
,\ C_4 = \py{r3(C_4 * 10**6)} \textup{ мкФ} \pm 10\%$$

\subsection*{Измерения процесса разряда конденсатора}

\begin{pycode}
import re
from tabulate import tabulate

def latex_table(data, headers):
  s = tabulate(data, headers=headers, tablefmt='latex')
  # https://github.com/gregbanks/python-tabulate/issues/5
  s = re.sub(r'(\^\{\})', "^", s); s = re.sub(r'\\([\$\_\{\}\^])', r'\1', s); s = re.sub(r'(\\textbackslash{})', r'\\', s)
  return s
\end{pycode}

\begin{pycode}
from numpy.polynomial.polynomial import polyfit

discharge_resistance_table = [
  [0, 1, 48, 17, 5],
  [10, 1, 45, 13, 5],
  [20, 1, 45, 10, 5],
  [30, 1, 44, 8, 5],
  [40, 1, 43, 6, 5],
  [50, 1, 40, 5, 5],
  [60, 1, 39, 9, 3],
  [70, 1, 37, 8, 3],
  [80, 1, 36, 7, 3],
  [90, 1, 34, 6, 3],
  [100, 1, 33, 8, 2],
  [200, 1, 24, 8, 1],
  [300, 1, 17, 3, 1],
  [400, 1, 12, 2, 1]
]

def log_decr(n, u0, un):
  return (1 / n) * np.log(u0 / un)
def q_factor(n, u0, un):
  return (2 * np.pi) / (1 - np.exp(-2 * log_decr(n, u0, un)))
def inductance(c, r_m, n, u0, un):
  return (np.pi**2 * c * (r_m + R_0)**2) / log_decr(n, u0, un)**2

def approx_resistance_at_log_decr(l):
  rs_ls = [[r, log_decr(n, u0, un)] for [r, _, u0, un, n] in discharge_resistance_table if r <= 100]
  rs, ls = np.array(rs_ls).T.tolist()
  r_b, r_a = polyfit(ls, rs, 1)
  return r_a*l + r_b

def approx_coeffs_r_l_dependence():
  rs_ls = [[r, log_decr(n, u0, un)] for [r, _, u0, un, n] in discharge_resistance_table if r <= 100]
  rs, ls = np.array(rs_ls).T.tolist()
  return polyfit(rs, ls, 1)

R_0 = -approx_resistance_at_log_decr(0)

table1 = [
  [r, t, u0, un, n, log_decr(n, u0, un), q_factor(n, u0, un), r3(inductance(C_1, r, n, u0, un) * 10**3) if r <= 100 else '-']
  for [r, t, u0, un, n] in discharge_resistance_table
]
\end{pycode}

\py{latex_table(table1, ['$R_{M}$, Ом', '$T$, мс', '$2U_i$, дел', '$2U_{i+n}$, дел', '$n$', '$\\lambda$', '$Q$', '$L$, мГн'])}

\subsection*{График зависимости $\lambda(R_M)$}

\begin{pycode}
import matplotlib.pyplot as plt

l_approx_range = np.arange(0, 400, step=1)
l_b, l_a = approx_coeffs_r_l_dependence()
l_approx = [l_a*r + l_b for r in l_approx_range]

rs, ls = np.array([[r, log_decr(n, u0, un)] for [r, _, u0, un, n] in discharge_resistance_table]).T.tolist()
plt.figure(figsize=(7, 4))
plt.plot(rs, ls, 'o')
plt.plot(l_approx_range, l_approx, '--')
plt.axis((0, 420, 0, 2.1))
plt.grid()
plt.xlabel('Сопротивление магазина $R_m$, Ом', fontsize=12)
plt.ylabel('Логарифмический декремент $\lambda$', fontsize=12)
plt.savefig('rl_plot.pdf', bbox_inches='tight')
\end{pycode}

\begin{figure}[H]
\includegraphics[width=0.65\textwidth]{rl_plot.pdf}
\end{figure}

\subsection*{Определение индуктивности контура $L$}

Определим собственное сопротивление контура $R_0$ как модуль абсциссы точки с $\lambda = 0$
аппроксимирующей прямой $\lambda(R_M)$:

$$R_0 = \py{r2(R_0)} \textup{ Ом}$$

Считая полное сопротивление контура $R = R_0 + R_M$, вычислим значение индуктивности
контура $L$ для малых значений сопротивления магазина $R_M \leq 100 \textup{ Ом}$ как:

$$L = \frac{\pi^2 C R^2}{l^2}$$

Усредним полученные результаты:

\begin{pycode}
L_mean = np.mean(np.array(table1)[:11, 7].astype(float))
\end{pycode}

$$L_{ср} = \py{r3(L_mean)} \textup{ мГн}$$

\subsection*{Период колебаний контура}

Найдем период колебаний в контуре при минимальном сопротивлении магазина
$R_M = 0 \textup{ Ом},\ R = R_0 = \py{r2(R_0)} \textup{ Ом}$ как: 

\begin{pycode}
T_0 = (np.pi * 2) / np.sqrt((1 / (L * C_1)) - (R_0**2 / (4 * L**2)))
\end{pycode}

$$T = \frac{2\pi}{\sqrt{\frac{1}{LC} - \frac{R^2}{4L^2}}} = \py{r2(T_0 * 10**3)}$$

\end{document}
