\documentclass[12pt, a4paper]{article}
\usepackage[a4paper, includeheadfoot, mag=1000,
            left=2cm, right=1.5cm, top=1.5cm, bottom=1.5cm,
            headsep=0.8cm, footskip=0.8cm]{geometry}
% Fonts
\usepackage{fontspec, unicode-math}
\setmainfont[Ligatures=TeX]{CMU Serif}
\setmonofont{CMU Typewriter Text}
\usepackage[english, russian]{babel}
% Indent first paragraph
\usepackage{indentfirst}
\setlength{\parskip}{5pt}
% Diagrams
\usepackage{graphicx}
\usepackage{floatrow}
% Python
\usepackage{pythontex}

\begin{document}
\begin{titlepage}

\noindent\textsc{ФГАОУ ВО «Санкт-Петербургский национальный исследовательский
университет информационных технологий, механики и оптики»\\[4mm]
Кафедра физики}
\vfill
\noindent\textbf{РАБОЧИЙ ПРОТОКОЛ И ОТЧЕТ\\[2mm]
ПО ЛАБОРАТОРНОЙ РАБОТЕ №1\\[4mm]
Изучение картины эквипотенциальных поверхностей и силовых линий
электростатического поля с помощью электролитической ванны }\\[16mm]
Преподаватель Ефремова Екатерина Александровна\\[2mm]
Студенты Лабушев Тимофей Михайлович, Нестеров Дали Константинович\\[2mm]
Группа P3202
\vfill
\noindent Санкт-Петербург\\[2mm]
2018 г.

\end{titlepage}

\section*{Введение}

\subsection*{Цель работы}

Построение сечений эквипотенциальных поверхностей и силовых линий
электростатического поля на основе экспериментального моделирования
распределения потенциала в слабопроводящей среде.

\subsection*{Задачи, решаемые при выполнении работы}

\begin{itemize}
\item Получение картины эквипотенциальных поверхностей
\item Оценка величины напряженности электрического поля в некоторых точках
\end{itemize}

\subsection*{Объект исследования}

Электрическое поле

\subsection*{Метод экспериментального исследования}

Прямые и косвенные измерения, графическое отображение

\subsection*{Требуемое оборудование и материалы}

\begin{enumerate}
\item Блок амперметра-вольтметра АВ1
\item Генератор напряжений ГН1
\item Электролитическая ванна ЭВ01
\item Проводящие тела разной формы
\item Проводники Ш4/Ш1,6: 4 шт
\end{enumerate}

\subsection*{Рабочие формулы и исходные формулы}

Структуру электростатического поля можно представить графически с
помощью силовых линий и с помощью эквипотенциальных поверхностей (на
чертеже обычно изображают сечения таких поверхностей).

Силовые линии строятся так, чтобы касательная в каждой точке такой
линии совпадала с вектором $\vec{E}$ электрической напряженности в этой же точке.
Густота силовых линий качественно характеризует величину напряженности
электростатического поля: в области, где линии идут гуще — модуль вектора $\vec{E}$
больше. Поскольку вектор электрической напряженности в данной точке
пространства определяет величину и направление силы действующей на любой
точечный заряд, помещенный в данную точку, этот вектор называют силовой
характеристикой электростатического поля.
Другая, энергетическая, характеристика электростатического поля —
потенциал $\varphi$. Работа сил электростатического поля над зарядом при
перемещении этого заряда по произвольной траектории определяется
разностью потенциалов между началом и концом траектории. Поверхности
равного потенциала $\varphi = const$ называются эквипотенциальными поверхностями.
Силовая и энергетическая характеристики электростатического поля
связаны друг с другом соотношением:
\begin{equation}
\vec{E} = -\nabla\varphi
\end{equation}

Градиент потенциала, стоящий в правой части этого соотношения в декартовой
системе координат определяется формулой
\begin{equation}
\nabla\varphi = i \frac{\partial\varphi}{\partial x} + j \frac{\partial\varphi}
+ k \frac{\partial\varphi}{\partial z}
\end{equation}
где $x,y,z$ — координаты, $\vec{i}, \vec{j}, \vec{k}$ — орты координатных осей.

Из свойств градиента следует, что:
\begin{itemize}
\item силовые линии пересекают эквипотенциальные поверхности под прямым углом
\item вектор напряженности направлен в сторону убывания потенциала
\item модуль вектора напряженности определяет пространственную быстроту
убывания потенциала в направлении силовой линии
\end{itemize}

Последнее означает, что, если известны потенциалы $\varphi_1$ и $\varphi_2$
двух точек, лежащих на одной силовой линии (см.рис.1), то средняя напряженность
между этими точками вычисляется по формуле.
\begin{equation} \label{eq:potdiff}
E_{12} = \frac{\varphi_1 - \varphi_2}{l_{12}}
\end{equation}

где $l_{12}$ – длина участка силовой линии между точками,
$l_{12} = \sqrt{(x_1 - x_2)^2 + (y_1 - y_2)^2}$. Если относительное
изменение локального значения напряженности между выбранными точками
невелико, то формула (3) дает значение близкое к напряженности на середине
участка 1-2.

\begin{figure}[H]
\floatbox[{\capbeside\thisfloatsetup{capbesideposition={right,top},capbesidewidth=4cm}}]{figure}[\FBwidth]
{\caption{$AA'$ — эквипотенциальная поверхность с потенциалом $\varphi_1$,
$BB'$ — с потенциалом $\varphi_2$, 1 и 2 – две точки одной силовой линии}}
{\includegraphics[width=0.3\textwidth]{lab1.png}}
\end{figure}

\newpage
\section*{Ход работы}

\begin{pycode}
from measurements import print_table, load_data_0, load_data_1, load_data_2
\end{pycode}

\subsection*{Измерения с двумя проводниками}

\pyc{print_table(load_data_0())}

\subsection*{Измерения с двумя проводниками и кольцом}

\begin{table}[H]
\resizebox{1.02\textwidth}{!}{
\pyc{print_table(load_data_1())}}
\end{table} 

\subsection*{Измерения с двумя проводниками и вторым проводящим телом}

\begin{table}[H]
\resizebox{1.02\textwidth}{!}{
\pyc{print_table(load_data_2())}}
\end{table} 

\subsection*{Графики, построенные по измерениям}

\begin{pycode}
import matplotlib.pyplot as plt
import numpy as np

fig, (ax0, ax1, ax2) = plt.subplots(3, figsize=(10, 20))

def plot_data(ax, data, lines):
  (xs, ys, vs) = data
  ax.axis((2, 28, 2, 24))
  ax.axis('equal')
  ax.tricontour(xs, ys, vs, lines, linewidths=0.5, colors='k')
  ax.grid(linestyle='--')
  
plot_data(ax0, load_data_0(), 6)
plot_data(ax1, load_data_1(), 15)
plot_data(ax2, load_data_2(), 17)

ax1.add_artist(plt.Circle((15, 10), radius=5, color='black'))

polygon_pts = np.array([(14, 7), (11, 10), (14, 13), (19, 13), (16, 10), (19, 7), (14, 7)])
ax2.add_artist(plt.Polygon(polygon_pts, color='black'))

plt.setp((ax1, ax2), xticks=list(range(2, 30, 2)), yticks=list(range(2, 20, 2)))
plt.savefig('plot.pdf', bbox_inches='tight')
\end{pycode}

\begin{figure}[H]
\includegraphics[width=0.65\textwidth]{plot.pdf}
\end{figure}

\subsection*{Расчеты}

Величина напряженности электрического поля оценивается в трех произвольно
взятых точках, для которых сняты измерения во всех опытах.

Для каждой из точек берется смежная ей по оси $x$, и по формуле (\ref{eq:potdiff})
находится проекция средней напряженности на эту ось. То же самое повторяется
для оси $y$.

Из полученных $x$ и $y$ компонент получаем полную среднюю напряженность,
выполнив их векторное сложение.

\begin{sympycode}
E_variables = symbols('\\varphi_1 \\varphi_2 \\varphi_3 l_{12} l_{13}')
phi1, phi2, phi3, l12, l13 = E_variables
Ex = (phi1 - phi2) / l12
Ey = (phi1 - phi3) / l13
E = sqrt(Ex**2 + Ey**2)
\end{sympycode}

$$E_x=\sympy{Ex},\ E_y=\sympy{Ey},\ E=\sympy{E}$$

Погрешность расчетов можно высчитать по следующей формуле:

\begin{sympycode}
lab_equip_delta = { phi1: 0.1, phi2: 0.1, phi3: 0.1, l12: 0.001, l13: 0.001 }
deltaE = sum(abs(diff(E, v) * lab_equip_delta[v]) for v in E_variables)

def pdE(wrt): return '\\frac{\\partial E}{\\partial ' + f'{wrt}' + '}'
latex_dE = "\\Delta E = " + " + ".join([pdE(wrt=v) + f'\\Delta {v}' for v in E_variables])
latex_partials = '$$$$'.join([pdE(wrt=v) + ' = ' + latex(diff(E, v)) for v in E_variables])
\end{sympycode}

$$\sympy{latex_dE}$$

$$\sympy{latex_partials}$$

Приборная погрешность составляет:
$$\Delta\varphi = 0.1, \Delta l = 0.001$$.

\noindent\textbf{Расчеты для первого опыта:}\\

\noindent
\begin{sympycode}
from measurements import comp_prereqs

(pts, xymap0, xymap1, xymap2) = comp_prereqs()

def render_xymap(xymap):
  for (x1, y1, x2, y2) in pts:
    var_subs = [(phi1, xymap[str(x1)][str(y1)]),
                (phi2, xymap[str(x1)][str(y2)]),
                (phi3, xymap[str(x2)][str(y1)]),
                (l12, (y1 - y2) / 100),
                (l13, (x2 - x1) / 100)]
    e = E.subs(var_subs)
    delta = deltaE.subs(var_subs)
    print('$E_{x_1 = ' + f'{x1:.0f}' + ',\ x_2 = ' + f'{x2:.0f}' + ',\ y_1 = ' + f'{y1:.0f}' + ',\ y_2 = ' + f'{y2:.0f}' + '} = '
          + latex(Float(e, 3)) + ',\ \\Delta E = ' + latex(Float(delta, 3)) + '$\\\\')

render_xymap(xymap0)
\end{sympycode}

\noindent\textbf{Расчеты для второго опыта:}\\

\noindent
\begin{sympycode}
render_xymap(xymap1)
\end{sympycode}

\noindent\textbf{Расчеты для третьего опыта:}\\

\noindent
\begin{sympycode}
render_xymap(xymap2)
\end{sympycode}

\section*{Вывод}

В ходе выполнения лабораторной работы нами были начерчены сечения
эквипотенциальных поверхностей, с помощью которых мы построили силовые линии
электростатического поля. Полученные картины позволили сделать нам вывод о том,
что около фигур густота силовых линий, которая характеризует величину напряженности
электростатического поля, увеличивается, из чего следует, что увеличивается и
величина напряженности.

На основе экспериментально полученных данных мы посчитали величину напряженности
в нескольких точках; расчеты подтвердили наше заключение, полученное по графикам:
величина напряженности увеличивается со сгущением силовых линий.

На картине также видно, что силовые линии примыкают к фигурам строго перпендикулярно,
что объясняется тем, что фигуры сами являются эквипотенциальными поверхностями.

\end{document}
