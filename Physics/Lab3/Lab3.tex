\documentclass[12pt, a4paper]{article}
\usepackage[a4paper, includeheadfoot, mag=1000,
            left=2cm, right=1.5cm, top=1.5cm, bottom=1.5cm,
            headsep=0.8cm, footskip=0.8cm]{geometry}
% Fonts
\usepackage{fontspec, unicode-math}
\setmainfont[Ligatures=TeX]{CMU Serif}
\setmonofont{CMU Typewriter Text}
\usepackage[english, russian]{babel}
% Indent first paragraph
\usepackage{indentfirst}
\setlength{\parskip}{5pt}
% Diagrams
\usepackage{graphicx}
\usepackage{floatrow}
% Python
\usepackage{pythontex}

\begin{document}
\begin{titlepage}

\noindent\textsc{ФГАОУ ВО «Санкт-Петербургский национальный исследовательский
университет информационных технологий, механики и оптики»\\[4mm]
Кафедра физики}
\vfill
\noindent\textbf{РАБОЧИЙ ПРОТОКОЛ И ОТЧЕТ\\[2mm]
ПО ЛАБОРАТОРНОЙ РАБОТЕ №3\\[4mm]
Исследование характеристик источника тока}\\[16mm]
Преподаватель Ефремова Екатерина Александровна\\[2mm]
Студенты Лабушев Тимофей Михайлович, Нестеров Дали Константинович\\[2mm]
Группа P3202
\vfill
\noindent Санкт-Петербург\\[2mm]
2018 г.

\end{titlepage}

\section*{Введение}

\subsection*{Цель работы}

Исследование зависимости полной мощности, полезной мощности, мощности
потерь, падения напряжения во внешней цепи и КПД источника от силы тока в
цепи

\subsection*{Задачи, решаемые при выполнении работы}

\begin{itemize}
\item Определение зависимости напряжения от силы тока
\item Определение ЭДС, внутреннего сопротивления источника и силы тока короткого замыкания
\item Определение мощностей и КПД, поиск сопротивления, при котором наблюдается наибольшая полезная мощност
\end{itemize}

\subsection*{Объект исследования}

Источник тока

\subsection*{Метод экспериментального исследования}

Прямые и косвенные измерения

\subsection*{Требуемое оборудование и материалы}

\begin{enumerate}
\item Генератор напряжений ГН1
\item Амперметр-вольтметр АВ1
\item Стенд с объектами исследования С3-ЭМ01
\item Проводники Ш4/Ш1,6-60 см (4 шт.), Ш4/Ш4-60 см (1 шт)
\end{enumerate}

\subsection*{Рабочие формулы и исходные формулы}

Если к источнику тока (см. рис. 1), обладающему внутренним
сопротивлением $r$, подключить внешнее сопротивление $R$, то напряжение на
зажимах источника $U$ согласно закону Ома для неоднородного участка цепи
можно представить в виде:
\begin{equation}
U = \varepsilon - Ir
\end{equation}

где $\varepsilon$ — электродвижущая сила источника (ЭДС); $I$ – сила тока,
текущего через источник.

\begin{figure}[H]
\floatbox[{\capbeside\thisfloatsetup{capbesideposition={right,top},capbesidewidth=4cm}}]{figure}[\FBwidth]
{\caption{Рабочая схема}}
{\includegraphics[width=0.25\textwidth]{lab3.png}}
\end{figure}

График зависимости напряжения $U$ от силы тока $I$ представляет собой прямую
(кривая \textit{а} на рис. 2). Пересечение с осью напряжений ($I = 0$)
происходит в точке $U = \varepsilon$, а точка пересечения графика с осью токов
($U = 0$) дает значение силы тока короткого замыкания источника $I_{sc}$.
Отметим, что последнее утверждение является идеализацией. В реальных источниках
ЭДС при токах близких к $I_{sc}$ линейный характер зависимости
напряжения $U$ от силы тока $I$ нарушается (кривая \textit{б} на рис. 2).
Это вызвано у одних источников уменьшением ЭДС при больших токах, у других
увеличением внутреннего сопротивления, у третьих одновременным влиянием двух этих причин.

\begin{figure}[H]
\floatbox[{\capbeside\thisfloatsetup{capbesideposition={right,top},capbesidewidth=4cm}}]{figure}[\FBwidth]
{\caption{График зависимости напряжения на источнике от силы тока в цепи}}
{\includegraphics[width=0.25\textwidth]{lab3_2.png}}
\end{figure}

\noindent Умножив обе части уравнения (1) на силу тока, протекающего по цепи, получим:
\begin{equation}
I\varepsilon = I^2R + I^2r
\end{equation}
Что можно представить в виде:
\begin{equation}
P = P_1 + P_2
\end{equation}

Здесь $P = I\varepsilon$ – полная мощность, развиваемая источником; $P_1 = I_2R = IU$ –
полезная мощность, т.е. мощность, развиваемая источником во внешней цепи
(на сопротивлении $R$); $P_2 = I_2r$ – потери мощности внутри источника (на
сопротивлении $r$).

Исследуем зависимость этих мощностей от силы тока. Графически (см.
рис. 3) зависимость полной мощности от силы тока P = I ε изображается
прямой линией, проходящей через начало координат. Полезная мощность из (2)
может быть представлена в виде:
\begin{equation}
P_1 = I\varepsilon − I^2r
\end{equation}

\begin{figure}[H]
\floatbox[{\capbeside\thisfloatsetup{capbesideposition={right,top},capbesidewidth=4cm}}]{figure}[\FBwidth]
{\caption{Зависимости полной мощности ($P$), полезной мощности ($P_1$) и мощности потерь ($P_2$)
от силы тока в замкнутой цепи}}
{\includegraphics[width=0.35\textwidth]{lab3_3.png}}
\end{figure}

Эта зависимость изображается параболой. Найдем значение тока $I_m$, при
котором полезная мощность максимальна. Для этого, взяв первую производную
от полезной мощности по силе тока, приравняем её к нулю:
\begin{equation}
\frac{dP_1}{dI} = \varepsilon - 2Ir = 0
\end{equation}

Откуда получим:
\begin{equation}
I_m = \frac{\varepsilon}{2r}
\end{equation}

Вторая производная $\frac{d^2P_1}{dI^2} = -2r$ отрицательна, так как при значении силы
тока $I_m$ полезная мощность имеет максимум $P_{1max}$. Подставляя выражение (6) в
уравнение (4) находим максимальное значение полезной мощности:
\begin{equation}
P_{1max} = \frac{\varepsilon^2}{4r}
\end{equation}

Заметим, что из уравнения (1) с учетом $U = IR$ следует, что
\begin{equation}
I = \frac{\varepsilon}{R + r}
\end{equation}

Сравнивая это выражение с выражением (6), видим, что при $I = I_m$ выполняется
равенство $R = r$. Следовательно, полезная мощность $P_1$ максимальна тогда,
когда сопротивление нагрузки равно внутреннему сопротивлению источника
питания $R = r$.

Потери мощности в источнике зависят от силы тока по формуле
\begin{equation}
P_2 = I^2r
\end{equation}

Графически эта зависимость (см. рис. 3) – парабола с вершиной в начале
координат и ветвями направленными вверх. При физически осмысленных
значениях $r > 0$, мы видим правую ветвь этой параболы.

Коэффициентом полезного действия (КПД) $\eta$ источника тока называется
величина, равная отношению полезной мощности к полной мощности,
затрачиваемой источником:
\begin{equation}
\eta = \frac{P_1}{P} = \frac{IU}{I\varepsilon} = \frac{U}{\varepsilon}
\end{equation}

Представив выражение для $U$ из (1) в (10), находим зависимость КПД от силы тока:
\begin{equation}
\eta = \frac{\varepsilon - Ir}{\varepsilon} = 1 - I\frac{r}{\varepsilon}
\end{equation}

Из уравнения (11) видно, что зависимость КПД от силы тока в цепи
изображается прямой линией (см. рис. 4), убывающей от значения $\eta = 1$ , при
токе $I = 0$, до значения $\eta = 0$, при силе тока
\begin{equation}
I_k = \frac{\varepsilon}{r}
\end{equation}

\begin{figure}[H]
\floatbox[{\capbeside\thisfloatsetup{capbesideposition={right,top},capbesidewidth=4cm}}]{figure}[\FBwidth]
{\caption{Зависимость КПД источника от силы тока в замкнутой цепи}}
{\includegraphics[width=0.3\textwidth]{lab3_4.png}}
\end{figure}

Это значение тока – уже упомянутый выше ток короткого замыкания.
Действительно, при $R = 0$ («короткое замыкание» источника, при этом $U = 0$) из
(1) видно, что сила тока достигает наибольшего значения, даваемого формулой
(12). Полезная мощность $P_1$ при этом убывает до нуля (см. рис. 3), так как при
сопротивлении $R = 0$ получаем $P_1 = I_{sc}U = I_{sc}^2R = 0$.

Полная мощность источника $P = I_{sc}\varepsilon$ и потери мощности
$P_2 = I_{sc}^2r$ при токе короткого замыкания $I = I_{sc}$ достигают
наибольшего значения и равны друг другу:

\begin{equation}
P_{max} = P_{2max} = \frac{\varepsilon^2}{r}
\end{equation}

Найдем значение КПД и соотношения между мощностями $P$, $P_1$, $P_2$ при
максимуме полезной мощности $P_1 = P_{1max}$. Полезная мощность максимальна
при условии $R = r$, КПД (10) при этом равен

\begin{equation}
\eta = \frac{U}{\varepsilon} = \frac{IR}{I(R+r)} = 0.5 = 50\%
\end{equation}

Отсюда, при токе $I = I_m$, полезная максимальная мощность равна
$P_{1max} = 0.5 P$ и с учетом (3) полезная мощность равна мощности потерь:
$P_{1max} = P_2$.

Из графиков зависимостей мощностей и КПД от силы тока (рис. 3, 4) видим,
что условия получения наибольшей полезной мощности $P_{1max}$ и наибольшего
КПД несовместимы. Когда $P_1$ достигает наибольшего значения, сила тока равна
$I_m$ и $\eta = 0.5$ или $50\%$. Когда же КПД близок к единице, полезная
мощность $P_1$ мала по сравнению с максимальной мощностью $P_{1max}$,
которую мог бы развить данный источник.

Прямолинейный характер зависимости напряжения $U$ от силы тока $I$ (см
формулу (1) и рис. 2), позволяет следующим, так называемым, методом
«короткого замыкания и холостого хода», определить параметры источника.
Изменяя в некоторых пределах сопротивление $R$, измеряют соответствующие
значения силы тока и напряжения. По измеренным значениям строят
прямолинейную зависимость $U$ от $I$. Продолжив ее до пересечения с осью
напряжений, находят напряжение «холостого хода» $U_х = \varepsilon$,
а продолжив до пересечения с осью токов, находят ток короткого замыкания
$I_{sc}$. Внутреннее сопротивление источника ЭДС определяют после
этого по формуле

\begin{equation}
r = \frac{\varepsilon}{I_{sc}}
\end{equation}

\section*{Результаты прямых измерений}

\begin{pycode}
import numpy as np
from tabulate import tabulate
import re

rs = np.arange(100, 1600, 100)
i_u_100ohms_to_1500 = np.array([
  [ma / 1000 for ma in [22.3, 19.1, 16.2, 14.4, 13.0, 11.7, 10.8, 10.0, 9.4, 8.7, 8.1, 7.7, 7.3, 6.9, 6.7]],
  [0.1, 2.12, 4.06, 5.3, 6.19, 7.06, 7.66, 8.24, 8.6, 9.11, 9.49, 9.77, 9.99, 10.3, 10.39]
])
i_u_1500ohms_to_100 = np.array([
  [ma / 1000 for ma in [22.3, 18.5, 16.3, 14.5, 13.3, 11.9, 10.7, 10.0, 9.3, 8.7, 8.1, 7.8, 7.3, 6.9, 6.7]],
  [0.01, 2.54, 3.98, 5.17, 5.98, 6.97, 7.76, 8.2, 8.65, 9.11, 9.45, 9.7, 10.02, 10.31, 10.39]
])
i_u_mean = (i_u_100ohms_to_1500 + i_u_1500ohms_to_100) / 2

def latex_table(data, headers):
  s = tabulate(data, headers=headers, tablefmt='latex')
  # https://github.com/gregbanks/python-tabulate/issues/5
  s = re.sub(r'(\^\{\})', "^", s); s = re.sub(r'\\([\$\_\{\}\^])', r'\1', s); s = re.sub(r'(\\textbackslash{})', r'\\', s)
  return s

data = np.array([i_u_100ohms_to_1500, i_u_100ohms_to_1500, i_u_mean])
data = data.reshape(-1, data.shape[-1]).transpose()
data = np.insert(data, 0, rs, axis=1).tolist()
\end{pycode}

\begin{table}[H]
\py{latex_table(data, ['$R$, Ом', '$I_1$, А', '$U_1$, В', '$I_2$, А', '$U_2$, В', '$I_{mean}$, А', '$U_{mean}$, В'])}
\end{table} 

Показания $I_1$ и $U_1$ были сняты при изменении сопротивления $R$ от 100 Ом
до 1500 Ом, $I_2$, $U_2$ — от 1500 Ом до 100 Ом.

\section*{Ход работы}

\subsection*{Зависимость $U(I)$}

\begin{pycode}
import matplotlib.pyplot as plt

currents1, voltages1 = i_u_100ohms_to_1500
currents2, voltages2 = i_u_1500ohms_to_100

plt.figure(figsize=(12,8))
plt.grid()
plt.xlabel('$I$, А', fontsize=14)
plt.ylabel('$U$, В', fontsize=14)
plt.plot(currents1, voltages1, label='От 100Ом к 1500Ом')
plt.plot(currents2, voltages2, label='От 1500Ом к 100Ом')
plt.legend()
plt.savefig('i_u_plot.pdf', bbox_inches='tight')
\end{pycode}

\begin{figure}[H]
\includegraphics[width=\textwidth]{i_u_plot.pdf}
\end{figure}

\subsection*{Определение внутреннего сопротивления}

Экстраполировав зависимость $U(I)$ до пересечения с осями координат,
найдем значения ЭДС $\varepsilon$ (при $I = 0$) и силы тока
короткого замыкания $I_{sc}$ (при $U = 0$) источника:

\begin{pycode}
from scipy.interpolate import interp1d

currents_mean, voltages_mean = i_u_mean
at_i_0 = interp1d(currents_mean, voltages_mean, fill_value = 'extrapolate')(0)
at_u_0 = interp1d(voltages_mean, currents_mean, fill_value = 'extrapolate')(0)

emf = np.round(at_i_0, 3)
isc = np.round(at_u_0, 3)

internal_resistance = np.round(emf / isc, 2)
\end{pycode}

\noindent
$\varepsilon = \py{emf}$ В, $I_{sc} = \py{isc}$ А.

Воспользовавшись формулой (15), рассчитаем внутренее сопротивление источника:

\noindent
$r = \frac{\varepsilon}{I_{sc}} = \py{internal_resistance}$ Ом

\subsection*{Определение мощности и КПД}  

Вычислим полную мощность $P$, развиваемую источником, полезную мощность $P_{net}$,
развиваемую источником во внешней цепи, потери мощности $P_{loss}$ внутри
источника и коэффициент полезного действия $\eta$ источника
по формулам (2), (4), (9) и (11) соответственно:

\begin{pycode}
gross_power = lambda i: np.round(net_power(i) + power_loss(i), 4)
power_loss = lambda i: max(0, np.round(i**2 * internal_resistance, 4))
net_power = lambda i: max(0, np.round(i * emf - power_loss(i), 4))
efficiency = lambda i: max(0, np.round(1.0 - i*(internal_resistance / emf), 3))

power_table = [[i, gross_power(i), net_power(i), power_loss(i), efficiency(i)] for i in i_u_mean[0]]
\end{pycode}

\begin{table}[H]
\py{latex_table(power_table, ['$I_{mean}$, А', '$P$, Вт', '$P_{net}$, Вт', '$P_{loss}$, Вт', "$\eta$"])}
\end{table} 

\subsection*{Зависимости $P(I)$, $P_{net}(I)$, $P_{loss}(I)$}

\begin{pycode}
i_mean, gross_power, net_power, power_loss, efficiency = np.array(power_table).transpose()

plt.figure(figsize=(12,8))
plt.grid()
plt.xlabel('$I$, А', fontsize=14)
plt.ylabel('$P$, Вт', fontsize=14)
plt.plot(i_mean, gross_power, label='$P$')
plt.plot(i_mean, power_loss, label='$P_{loss}$')
plt.plot(i_mean, net_power, label='$P_{net}$')
plt.legend()
plt.savefig('power_plot.pdf', bbox_inches='tight')
\end{pycode}

\begin{figure}[H]
\includegraphics[width=\textwidth]{power_plot.pdf}
\end{figure}
\newpage

Найдем точку $I_m$, в которой $P_1$ принимает максимальное значение:

\begin{pycode}
max_p_i, max_p, max_p_r = max(zip(currents_mean, net_power, rs),
key=lambda z: z[1])
\end{pycode}

\noindent
$I_m = \py{np.round(max_p_i, 4)}$ А, $P_{1max} = \py{np.round(max_p, 4)}$ Вт

Точке $I_m = \py{np.round(max_p_i, 4)}$ А соотвествует $R = \py{max_p_r}$ Ом.

Теоретические значения, рассчитанные по формулам (6) и (7) соответственно:

\noindent
$I_{m_t} = \py{np.round(emf / (2*internal_resistance), 4)}$ А,
$P_{1max_t} = \py{np.round(emf**2 / (4*internal_resistance), 4)}$ Вт

На пересечении прямой $P_1(I)$ с осью $I$ ($P_1 = 0$) мы получаем $I_{sc} =
\py{interp1d(net_power, currents_mean, kind = 'quadratic', fill_value = 'extrapolate')(0)}$ А.

\subsection*{Зависимость $\eta(I)$}

\begin{pycode}
plt.figure(figsize=(12,8))
plt.grid()
plt.xlabel('$I$, А', fontsize=14)
plt.ylabel('$\eta$', fontsize=14)
plt.plot(i_mean, efficiency)
plt.savefig('efficiency_plot.pdf', bbox_inches='tight')
\end{pycode}

\begin{figure}[H]
\includegraphics[width=\textwidth]{efficiency_plot.pdf}
\end{figure}

\begin{pycode}
closest_eta_i, closest_eta = min(zip(currents_mean, efficiency),
key=lambda z: abs(0.5 - z[1]))
\end{pycode}

В ближайшей к $\eta = 0.5$ точке $I_m = \py{np.round(closest_eta_i, 4)}$ А,
что соответствует значению, полученному на предыдущем этапе работы.


\section*{Вывод}

\end{document}
