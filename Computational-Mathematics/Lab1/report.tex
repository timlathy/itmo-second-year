\documentclass[listings]{labreport}
\departmentsubject{Кафедра вычислительной техники}{Вычислительная математика}
\titleparts{Лабораторная работа №1}{Численные методы решения нелинейных уравнений}
\students{Лабушев Тимофей Михайлович}

\begin{document}

\maketitlepage

\section*{Цель работы}

Программно реализовать метод половинного деления и метод Ньютона для численного
решения нелинейных уравнений; применить программу для решения заданного кубического уравнения.

\section*{Порядок выполнения работы}

В ходе выполнения работы рисуется график функции, строятся таблицы работы
методов для нахождения двух корней (крайнего справа и крайнего слева на графике).

\section*{Рабочие формулы методов}

\textbf{Метод половинного деления:} $$x_n = \frac{a_n + b_n}{2}$$

где $a_n, b_n$ — границы интервала.\\[4mm]

\textbf{Метод Ньютона} $$x_n = x_{n-1} - \frac{f(x_{n-1})}{f'(x_{n-1})}$$

\section*{Программная реализация}

\subsection*{Метод половинного деления}
\lstinputlisting[firstline=5, lastline=26, basicstyle=\scriptsize]{bisect.c}

\subsection*{Метод Ньютона}
\lstinputlisting[firstline=3, lastline=23, basicstyle=\scriptsize]{newton.c}

\subsection*{Вспомогательные функции}
\lstinputlisting[firstline=4, lastline=21, basicstyle=\scriptsize]{primitives.h}

\newpage
\section*{Расчет корней функции}

Посчитаем приближенные (с точностью $10^{-6}$) корни уравнения, поочередно
вводя в программу следующие изолирующие интервалы: $[-4, -1.8],\ [-1.8, -0.8],\ [0.5, 3]$.

\begin{verbatim}
[-4, -1.8]:
Метод половинного деления:
  число итераций = 22, x = -2.409752893447876, f(x) = -1.1497809255800462e-06
Метод Ньютона:
  число итераций = 6, x = -2.4097526379051217, f(x) = -2.850768510143098e-10

[-1.8, -0.8]:
Метод половинного деления:
  число итераций = 19, x = -1.2100894927978516, f(x) = 7.745036527673221e-07
Метод Ньютона:
  число итераций = 3, x = -1.2100892127038232, f(x) = -8.232042780775828e-08

[0.5, 3]:
Метод половинного деления:
  число итераций = 22, x = 1.3398419618606567, f(x) = 8.070110228963756e-07
Метод Ньютона:
  число итераций = 5, x = 1.339841911232238, f(x) = 3.229421552397582e-07
\end{verbatim}

\section*{Вывод}

В ходе выполнения лабораторной работы я познакомился с двумя итерационными
методами решения нелинейных функций: методом половинного деления отрезка и
методом Ньютона. Первый требует б\textit{о}льших вычислительных затрат для
получения результата, так как имеет линейную сходимость по сравнению с
квадратичной у метода Ньютона. Измерение скорости выполнения методов показало
различие в 40\%, однако порядок затраченного времени составлял сотни
наносекунд, что делает разницу незначительной.

Метод Ньютона также требователен к выбору первоначального приближения, 
в то время как методов половинного деления получает результат на изолирующем
интервале любой длины (с соответствующем увеличением времени выполнения).

\end{document}
