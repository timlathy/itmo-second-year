\documentclass[listings]{labreport}
\departmentsubject{Кафедра вычислительной техники}{Вычислительная математика}
\titleparts{Лабораторная работа №1}{Численные методы решения нелинейных уравнений}
\students{Лабушев Тимофей Михайлович}

\begin{document}

\maketitlepage

\section*{Цель работы}

Программно реализовать метод половинного деления и метод Ньютона для численного
решения нелинейных уравнений; применить программу для решения заданного кубического уравнения.

\section*{Порядок выполнения работы}

В ходе выполнения работы рисуется график функции, строятся таблицы работы
методов для нахождения двух корней (крайнего справа и крайнего слева на графике).

\section*{Рабочие формулы методов}

\textbf{Метод половинного деления:} $$x_n = \frac{a_n + b_n}{2}$$

где $a_n, b_n$ — границы интервала.\\[4mm]

\textbf{Метод Ньютона} $$x_n = x_{n-1} - \frac{f(x_{n-1})}{f'(x_{n-1})}$$

\section*{Программная реализация}

\subsection*{Метод половинного деления}
\lstinputlisting[firstline=5, lastline=26, basicstyle=\scriptsize]{bisect.c}

\subsection*{Метод Ньютона}
\lstinputlisting[firstline=3, lastline=23, basicstyle=\scriptsize]{newton.c}

\subsection*{Вспомогательные функции}
\lstinputlisting[firstline=4, lastline=21, basicstyle=\scriptsize]{primitives.h}

\end{document}
