\documentclass[12pt,a4paper]{report}
\usepackage[a4paper, mag=1000, left=2.5cm, right=1cm, top=2cm, bottom=2cm, headsep=0.7cm, footskip=1cm]{geometry}
% Fonts
\usepackage{fontspec, unicode-math}
\setmainfont[Ligatures=TeX]{CMU Serif}
\setmonofont{CMU Typewriter Text}
\usepackage[english, russian]{babel}
% Indent first paragraph
\usepackage{indentfirst}
\setlength{\parskip}{5pt}
% Code listings
\usepackage[dvipsnames]{xcolor}
\usepackage{listings}

\begin{document}

\begin{titlepage}
\begin{center}

\textsc{ФГАОУ ВО «Санкт-Петербургский национальный исследовательский университет информационных технологий, механики и оптики»\\[4mm]
Кафедра вычислительной техники}
\vfill
\textbf{ЛАБОРАТОРНАЯ РАБОТА №1\\[4mm]
Вычислительная математика}\\[16mm]
Лабушев Тимофей Михайлович
\\[2mm]Группа P3202
\vfill
Санкт-Петербург\\[2mm]
2018 г.
\end{center}
\end{titlepage}

\section*{Цель работы}

Программно реализовать метод половинного деления и метод Ньютона для численного
решения нелинейных уравнений; применить программу для решения заданного кубического уравнения.

\section*{Порядок выполнения работы}

В ходе выполнения работы рисуется график функции, строятся таблицы работы
методов для нахождения двух корней (крайнего справа и крайнего слева на графике).

\section*{Рабочие формулы методов}

\textbf{Метод половинного деления:} $$x_n = \frac{a_n + b_n}{2}$$

где $a_n, b_n$ — границы интервала.\\[4mm]

\textbf{Метод Ньютона} $$x_n = x_{n-1} - \frac{f(x_{n-1})}{f'(x_{n-1})}$$

\section*{Программная реализация}

\subsection*{Метод половинного деления}
\lstinputlisting[firstline=5, lastline=26, basicstyle=\scriptsize]{bisect.c}

\subsection*{Метод Ньютона}
\lstinputlisting[firstline=3, lastline=23, basicstyle=\scriptsize]{newton.c}

\subsection*{Вспомогательные функции}
\lstinputlisting[firstline=4, lastline=21, basicstyle=\scriptsize]{primitives.h}

\end{document}
