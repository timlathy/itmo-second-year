\documentclass[en, listings]{labreport}
\departmentsubject{Department of Computational Technologies}{System Programming Languages}
\titleparts{Lab Work \#3}{Scalar Product, Primality Test}
\students{Timothy Labushev}

\begin{document}

\maketitlepage

\section*{Summary}

This report covers an introductory assignment for C programming language. The
task is to implement two numerical routines: \textit{scalar (dot) product} and
\textit{primality test}.

\subsection*{Dot Product}

The \textit{dot product} of two vectors $a = (a_1, a_2, \dots, a_n)$
and $b = (b_1, b_2, \dots, b_n)$ is defined as:

$$\sum_{i = 1}^n a_i b_i = a_1 b_1 + a_2 b_2 + \dots + a_n b_n$$

The inputs are provided as two global arrays of \texttt{int}s defined
at compile time. A function that computes the dot product has no
side effects; it is called by the \texttt{main} function, and the result is
printed to \texttt{stdout}.

\subsection*{Primality Test}

A \textit{prime} is a natural number greater than 1 that cannot be formed
by multiplying two smaller natural numbers.

The primality test function has the following signature:
\texttt{int is\_prime (unsigned long n)}. It returns \texttt{1}
if \texttt{n} is a prime number, \texttt{0} otherwise.

The input is read from \texttt{stdin} by the \texttt{main} function.
To convert input to a number, the \texttt{scanf} function from the
C standard library is used.

\section*{Code Listing}

\subsection*{Entrypoint}

\lstinputlisting[firstline=5, basicstyle=\scriptsize]{main.c}

\subsection*{Dot Product}

\lstinputlisting[firstline=3, basicstyle=\scriptsize]{dot.c}

\subsection*{Primality Test}

\lstinputlisting[firstline=3, basicstyle=\scriptsize]{prime.c}

\section*{Lessons Learned}

While completing the assignment, I familiarized myself with the 
basics of C programming language, namely common syntactic constructs,
such as control flow statements and function definitions, 
and main types. I also learned to take into account a number of important
considerations that are not obvious when coming from higher-level languages.

For instance, array indexes and lengths have to be represented using
the \texttt{size\_t} type alias. This is in constrast to e.g. Java,
where the \texttt{List.size()} method returns an \texttt{int}.

In addition to that, while the use of a data immutability modifier,
\texttt{const}, is encouraged in most cases, it should be applied with care.
One of the problems that may be encountered is some external (library) functions
requiring mutable value pointers. While typecasting is possible, it will
result in a compiler warning, and with a reason: an attempt to modify the value
could result in a runtime error due to the data being stored in a read-only section.

\end{document}
