\documentclass[en, listings]{labreport}
\departmentsubject{Department of Computational Technologies}{System Programming Languages}
\titleparts{Lab Work \#4}{Linked List, Higher-Order Functions}
\students{Timothy Labushev}

\begin{document}

\maketitlepage

\section*{Summary}

This report covers an assignment on implementing a \textit{singly linked list}
using dynamic memory allocation in C programming language.

The final program has a command-line interface that allows the user to input
an arbitrary number of signed integers (alternatively, have them read from
a file), from which a linked list is constructed. The user is then presented
with miscellaneous statistics, such as list \textit{length}, \textit{sum} of its
elements, \textit{squares} and \textit{cubes} of its elements.

\section*{Code Listing}

Shown below are the function signatures of linked list manipulation routines.

The implementation is omitted from the printed report for brevity reasons.
It is available at \texttt{\small github.com/timlathy/itmo-second-year/tree/master/System-Programming-Languages/Lab4}.

\subsection*{Core List Functions}

\lstinputlisting[language=C, firstline=7, lastline=32, basicstyle=\scriptsize]{llist.h}

\subsection*{List Iteration Functions}

\lstinputlisting[language=C, firstline=6, lastline=17, basicstyle=\scriptsize]{llist_iter.h}

\subsection*{List IO Functions}

\lstinputlisting[language=C, firstline=8, lastline=17, basicstyle=\scriptsize]{llist_io.h}

\section*{Lessons Learned}

While completing the assignment, I familiarized myself with the 
basics of low-level dynamic memory management in C programming language.

I studied the implementation of general higher-order functions (\texttt{map},
\texttt{foreach}, \texttt{fold}) operating on a singly linked list.
I learned to take into account the limitations of function passing in C,
namely the lack of lexical closures: to simulate their behavior,
I had to pass the free variables of the callback function via a
\texttt{void*} pointer, which it then casts to the type it expects to receive.

\end{document}
