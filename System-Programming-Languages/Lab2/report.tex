\documentclass[12pt,a4paper]{report}
\usepackage[a4paper, mag=1000, left=2.5cm, right=1cm, top=2cm, bottom=2cm, headsep=0.7cm, footskip=1cm]{geometry}
% Fonts
\usepackage{fontspec, unicode-math}
\setmainfont[Ligatures=TeX]{CMU Serif}
\setmonofont{CMU Typewriter Text}
\usepackage[english, russian]{babel}
% Indent first paragraph
\usepackage{indentfirst}
\setlength{\parskip}{5pt}
% Code listings
\usepackage[dvipsnames]{xcolor}
\usepackage{listings}

\begin{document}

\begin{titlepage}
\begin{center}
\textsc{
  National Research University ITMO\\[4mm]
  Department of Computational Technologies
}
\vfill
\textbf{
  Lab Report \#2\\[2mm]
  System Programming Languages
  \\[20mm]
}
Timothy Labushev\\[2mm]
Group P3202
\vfill
Saint Petersburg\\[2mm]
2018
\end{center}
\end{titlepage}

\newpage
\section*{Summary}

This report covers an assignment on implementing a \textit{dictionary} in
x86-64 assembly — a linked list data structure where each element contains
a pointer to the next entry, a null-terminated string key, and an arbitrary value.

The dictionary is constructed statically. As a part of the assignment,
a special macro called \texttt{colon} is written to automate
the creation of the data structure. The macro accepts two arguments,
a null-terminated string key and an assembly label for the internal value.

Furthermore, a dictionary search routine (called \texttt{find\_word}) is created
to look up a value given its key. It accepts two arguments: a pointer to
a null-terminated string key and the \textit{dictionary head}, i.e. the pointer
to the last defined element. The routine returns a pointer to the found element
or 0 if there was no match.

Finally, a simple testing routine is defined. It reads up to 255 characters
from \texttt{stdin} and attempts to find a value for this key in the dictionary.
If found, the corresponding value is printed to \texttt{stdout}, otherwise
an error message is sent to \texttt{stderr}.

I/O is handled by special routines implemented as the first assignment
of the course.

\section*{Code Listing}

\subsection*{Word Definition Macro}

\lstinputlisting[firstline=3, basicstyle=\scriptsize]{colon.inc}

\subsection*{Dictionary Lookup}

\lstinputlisting[firstline=3, basicstyle=\scriptsize]{dict.asm}

\subsection*{Testing Routine}

\lstinputlisting[firstline=3, basicstyle=\scriptsize]{main.asm}

\section*{Lessons Learned}

While completing the assignment, I have familiarized myself with the different
stages of assembling a program.

The first task was to use the macro system of NASM to automate dictionary insertion.
To verify the correctness of my implementation, I had to run only the first stage of
the build — preprocessing — which expands macros and \texttt{\%include} directives by
performing source code substitution. This forced me to be particularly careful
with the syntax, as the proprocessor does not verify it.

Next, I had to split my source code into separate files, which are translated
into machine code separately as well. The resulting files are called \textit{object files}:
they consist of program text, static data, and a symbol table, which contains
references to symbols (memory addresses) from other files. I have used the \texttt{nm}
command to check the symbol table.

A binary executable is produced by the \texttt{ld} linker. It assigns absolute memory
locations (the object files contain relative offsets) and resolves symbol references.

\end{document}
