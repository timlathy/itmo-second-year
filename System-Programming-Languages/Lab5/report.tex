\documentclass[en, listings]{labreport}
\departmentsubject{Department of Computational Technologies}{System Programming Languages}
\titleparts{Lab Work \#5}{BMP Image Rotation}
\students{Timothy Labushev}

\begin{document}

\maketitlepage

\section*{Summary}

This report covers an assignment on implementing a program that rotates 24-bit
BMP images to 90 degrees clockwise.

The final program has a command-line interface that allows the user to input
the source and the destination files. I/O errors are handled separately from
file format parsing, i.e. upon receiving a path to a non-existent file, the
program generates a different error message than when a valid non-BMP file
is provided.

\section*{Code Listing}

\subsection*{Main File}

\lstinputlisting[language=C, firstline=6, basicstyle=\scriptsize]{main.c}

\subsection*{Image Rotation Function}

\lstinputlisting[language=C, firstline=6, lastline=15, basicstyle=\scriptsize]{image.h}

\noindent\rule{\textwidth}{1pt}\\

\lstinputlisting[language=C, firstline=4, basicstyle=\scriptsize]{image.c}

\subsection*{BMP Parsing Functions}

\lstinputlisting[language=C, firstline=7, lastline=47, basicstyle=\scriptsize]{image_bmp.h}

\noindent\rule{\textwidth}{1pt}\\

\lstinputlisting[language=C, firstline=5, basicstyle=\scriptsize]{image_bmp.c}

\section*{Lessons Learned}

While completing the assignment, I familiarized myself with the 
basics of binary input and output in C programming language.

I studied the use of structures for representing file headers, including the
use of compiler directives (\texttt{\_\_attribute\_\_((packed))}) to force the
required memory layout.

In addition to that, I learned to perform $90^\circ$ rotation on raster images.

\end{document}
