\documentclass[en, listings]{labreport}
\departmentsubject{Department of Computational Technologies}{System Programming Languages}
\titleparts{Lab Work \#6}{Dynamic Memory Allocator}
\students{Timothy Labushev}

\begin{document}

\maketitlepage

\section*{Summary}

This report covers an assignment on implementing a simple dynamic memory
allocator in C programming language.

The final program provides \texttt{heap\_alloc} and \texttt{heap\_free}
functions, which are similar to \texttt{malloc} and \texttt{free} from
the C standard library.

The heap consists of contiguous memory segments obtained from the OS
using the \texttt{mmap} system call. The segments are split into
chunks, which contain an allocated block of memory as well as metadata
required to manage the heap. Contiguous free space is represented
as chunks with the \texttt{is\_free} flag set to \texttt{true}.

The heap is thus viewed as a singly linked list of chunks of variable
size, which may be \textit{free} or \textit{allocated}.

\section*{Code Listing}

\lstinputlisting[language=C, firstline=13, basicstyle=\scriptsize]{alloc.c}

\section*{Lessons Learned}

While completing the assignment, I familiarized myself with the 
basics of dynamic memory management, as well as data alignment and padding.
I had to pay close attention to the memory layout of the heap, ensuring
that pointers returned by \texttt{heap\_alloc} are aligned on an eight-byte
boundary to improve performance.

Another challenge I faced was coalescing of successive free chunks:
one of the limitations of the architecture I've chosen (singly linked list)
is the inability to look at the preceding chunks when freeing memory.
The solution I've found was performing coalescing at allocation time,
which may hurt performance somewhat, but results in 8 bytes (pointer size)
saved per allocation.

\end{document}
